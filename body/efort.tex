% !TEX root =  ../main.tex
\subsection{Manipulability-based heuristics for locally-optimal contact selection}
The other kind of heuristics that we use are based on the analysis of the motion capabilities
provided by a candidate configuration regarding the task being achieved.
These heuristics allow to locally optimize a contact location with respect to a particular task.
For instance, one can be interested in configurations that allow to efficiently exert a force compatible with the global direction of motion,
a high velocity in a given direction, or to stay away from singular configurations.
In this section, we derive three such heuristics from the work on manipulability by \cite{Yoshikawa1984}, that we present first.

\subsubsection{The force and velocity ellipsoids.}
For a limb configuration $\mathbf{q}^k$, the Jacobian matrix
 $\mathbf{J}(\mathbf{q}^k)$   defines the relation:
 
\begin{equation} \label{jac}
\mathbf{\dot{p}^k} = \mathbf{J}(\mathbf{q}^k) \dot{\mathbf{q}}^k
\end{equation}

For clarity in the rest of the section we omit the $k$ indices and write $\mathbf{J}(\mathbf{q}^k)$ as $\mathbf{J}$.
 
As a linear approximation of the forward-kinematics function $f$ , $\mathbf{J}$ describes how small
variations from the configuration $\mathbf{q}$ affect the position vector $\mathbf{p}$.
It can be seen as the transformation mapping a velocity in the configuration space into a velocity in
the Euclidian space.

Now we consider the unit ball in the configuration space $C$ defined by the set of joint velocities
for which the norm is at most 1:
\begin{equation} \label{ball}
||\dot{\mathbf{q}}||^2 \leq 1
\end{equation}

From \eqref{jac} we can obtain the following equality: \adnote{how?}
\begin{equation} \label{jaceq}
\mathbf{\dot{p}}^T(\mathbf{J}\mathbf{J}^T)^{-1}\mathbf{\dot{p}} = \dot{\mathbf{q}}^T \dot{\mathbf{q}}
\end{equation}

We can use \eqref{jaceq} to map the ball into an ellipsoid in the Euclidian space $\mathbb{R}^m$:

\begin{equation} \label{ellipsoid}
\mathbf{\dot{p}}^T(\mathbf{J}\mathbf{J}^T)^{-1}\mathbf{\dot{p}} \leq 1
\end{equation}
This ellipsoid is called the manipulability ellipsoid, or velocity ellipsoid, introduced by \cite{Yoshikawa1984}. It describes the set of velocities that can
be reached under the constraint \eqref{ball} for the current configuration.
The longer an axis of the ellipsoid is, the faster the configuration can move along the direction of the axis.
Fig.~\ref{sec:efort_ellipsoid} - left shows the velocity ellipsoid for different configurations of a manipulator with two degrees of freedom.

\begin{figure}[htb]
  \centering
	\begin{overpic}[width=1\linewidth]{figures/EFORT/ellipsoid}
		\put (8.5,1.2) {\small{Velocity ellipsoid}}
		\put (62.5,1.2) {\small{Force ellipsoid} \tiny{(scale 0.5)}}
	\end{overpic}
  % \includegraphics[width=1\linewidth]{ressources/screens/sampling/canap/bd2_cmyk}
  \caption{Examples of velocity and force ellipsoids for a manipulator composed of 2 dofs and 2 segments.
Only the horizontal and vertical speeds are shown (not the rotation speeds), since it would require being able to draw in four dimensions.}
		   \label{sec:efort_ellipsoid}
\end{figure}


Similarly to the velocity ellipsoid, \citeauthor{Yoshikawa1984} also defines the force ellipsoid.
Considering: a force vector $\mathbf{f}$ expressed in the task space $\mathbb{R}^m$;
the equivalent joint torque vector $\mathbf{\tau}$;
we can define the mechanical work in both spaces:
\begin{equation*} \label{power}
\dot{\mathbf{q}}^T \mathbf{\tau} = \dot{\mathbf{p}}^T \mathbf{f}
\end{equation*}


Exploiting \eqref{jac}, we can easily see that the set of achievable forces in $\mathbb{R}^m$ subject to the constraint:
\begin{equation*} \label{ballforce}
||\mathbf{\tau}||^2 \leq 1
\end{equation*}
is the so-called force ellipsoid (Fig.~\ref{sec:efort_ellipsoid} - right), defined by:
\begin{equation} \label{ellipsoidforce}
\mathbf{f}^T (\mathbf{J}\mathbf{J}^T) \mathbf{f} \leq 1
\end{equation}

\subsubsection{Manipulability-based heuristics.}
From these definitions, we can derive three useful heuristics, that all account \replaced{for}{from} the environment and the task being performed.
The first one, EFORT, was introduced by \cite{Tonneau2014}; the other two are new minor contributions, derived from these previous works.

With EFORT, we define the efficiency of a configuration as the ability of a limb to exert a force in a given direction.
We thus consider the force ellipsoid as a basis for our heuristic.
In a given direction $\mathbf{m}$, the length of the ellipsoid is given by the force-transmission ratio \citep{1087795}:
\begin{equation*}
f_\mathsf{T}(\mathbf{q}, \mathbf{m}) = [\mathbf{m}^{T}(JJ^{T})\mathbf{m}]^{-\frac{1}{2}}
\end{equation*}

In our problem, to compare candidate configurations, we include the quality of the contact surface, and choose $\mathbf{m}$ as the direction
opposite to the local motion (thus given by the difference between two consecutive root positions):

\begin{equation}
h_{EFORT}(\mathbf{q}, \mathbf{m}) = [\mathbf{m}^{T}(\mathbf{J}\mathbf{J}^T)\mathbf{m}]^{-\frac{1}{2}} ( \mu_0 \mathbf{n}^T \mathbf{m})
\end{equation}
where $\mu_0$ and $\mathbf{n}$ are respectively the friction coefficient and the normal vector of the contact surface.


If the ability to generate large velocities at the effector is considered, we define a new heuristic $h_{vel}$ with a similar reasoning on the velocity ellipsoid:
\begin{equation}
h_{vel}(\mathbf{q}, \mathbf{m}) = [\mathbf{m}^{T}(\mathbf{J}\mathbf{J}^T)^{-1}\mathbf{m}]^{-\frac{1}{2}} ( \mu_0 \mathbf{n}^T \mathbf{m})
\end{equation}

$h_{EFORT}$ and $h_{vel}$ will favor contacts that allow large efforts or fast modifications in the velocity.
$EFORT$ in particular is useful for tasks such as standing up, pushing / pulling.
In other less demanding cases, manipulability can also be considered \deleted{regarding its original purpose, that means} to avoid singularities.
To do so, we can consider the manipulability measure $w$, also given by \citeauthor{Yoshikawa1984}:

\begin{equation*} \label{ellipsoid}
w(\mathbf{q}) = \sqrt{\mathbf{J}\mathbf{J}^T}
\end{equation*} \adnote{Didn't you forget something in this equation? I guess $w$ is a scalar, but the right side is a matrix.}
$w$ measures the ``distance'' between a given configuration and a singular configuration. When $w$ is equal to 0, the configuration is in a singular state;
the greater $w$ is, the further away the configuration is from singularity.
We then derive:
\begin{equation}
h_{w}(\mathbf{q}, \mathbf{m}) = \sqrt{\mathbf{J}\mathbf{J}^T} ( \mu_0 \mathbf{n}^T \mathbf{m})
\end{equation}

$h_{EFORT}$, $h_{vel}$ and $h_{w}$ define three kinematic heuristics, fast to compute (indeed, thanks to our sampling-based approach, the Jacobian and inverse products
can be precomputed offline), that allow the planner to select the \replaced{best}{more relevant}
candidates \replaced{with respect to}{regarding} the local motion task along the guide path.
This results in faster computation times, and a reduced number of overall contact switches.



