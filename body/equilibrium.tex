\section{Heuristics for relevant contact selection}

\subsection{A heuristic for robust quasi static balance}
\noindent\textbf{EFORT criterion:} If only relying on the random sampling to select new contacts, the planner produces inefficient postures. The resulting contact sequence is then poorly efficient and unnatural. Moreover, the limbs are not well configured and are not able to efficiently follow the general movement: contacts break frequently.

When creating additional contacts, we therefore propose to select particular configurations that allow to exert a force compatible with the direction of motion. This task efficiency is measured  with the Extended FORce Transmission ratio (EFORT)~\cite{Tonneau2014}.
%
The measure of EFORT is given by
\begin{equation}
\alpha_{EFORT}(\mathbf{q}^k, \mathbf{m}) = [\mathbf{m}^{T}(\mathbf{J}\mathbf{J}^T)\mathbf{m}]^{-\frac{1}{2}} ( \nu_0 \mathbf{n}^T \mathbf{m})
\end{equation}
where $\mathbf{J}$ is the Jacobian matrix of the limb $R^k$ in configuration $\mathbf{q}^k$; $\nu_0$ is the friction coefficient of the contact surface; $\mathbf{n}$ is the normal of the contact surface; and $\mathbf{m}$ is the direction opposite to the motion,
given by the 3D vector connecting $\mathbf{q}_{i}^0$ and $\mathbf{q}_{i+1}^0$.
%
The first part of the equation measures the ratio between the joint  torques and the resulting force applied along $\mathbf{m}$. The second part quantifies the odds of slipping while applying a force along $\mathbf{m}$. 

\noindent\textbf{Optimization at creation:} In practice, a database of configurations is stored for each limb, which can be considered as manipulator arms. The database is implemented  as an octree data structure, indexed by the end-effector positions of the configurations (and additionally storing $\mathbf{J}$). 
%Fig.~\ref{fig:contact_gen}-2 presents a few configurations for the right arm of the robot.  
Upon request, the octree returns a set of configurations close to contact (Fig.~\ref{fig:contact_gen}-3). These candidates are sorted based on their task efficiency, given by $\alpha_{EFORT}$. The first candidate in this list satisfying the balance criterion and is collision free is selected and projected on the contact surface using our inverse kinematics solver.



%Therefore to generate a contact, we want to find a configuration which is collision free, allows to maintain balance, and maximizes EFORT. We choose this configuration among a set of possible candidates, rapidly computed thanks to a contact generator, also proposed in~\cite{Tonneau2014}. We recall the overall idea here. We consider each limb of the robot as a manipulator arm attached to the root. We store a database of configurations for each manipulator. The database is implemented  as an octree data structure, indexed by the end-effector positions of the configurations. The jacobian matrix $\mathbf{J}$ of each configuration is also stored. Figure~\ref{fig:contact_gen} -- 2 presents a few configurations for the right arm of the robot.  


%To summarize: given a feasible trajectory for the root of the robot $R$, we are able to generate a discrete sequence of contact configurations  describing a feasible motion for $R$, between a start an goal configurations. The contact configurations are chosen to be statically balanced and task efficient, which means that they can generate a force  contributing to the motion of the root along the trajectory. 



