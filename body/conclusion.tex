 \section{Discussion and future work} 
\label{sec:conclusion}

In this paper we consider the \gls{cluttered} contact planning problem, formulated as two sub-problems that we address sequentially.
The first problem $\mathcal{P}_1$ consists in computing an \gls{equilibrium feasible} guide path for the root of the robot;
 the second problem $\mathcal{P}_2$ is the computation of a discrete sequence of whole-body configurations along the root path.

Our contribution to \Pa is the introduction of a low-dimensional space $C_{reach}$, an approximation of the space of \gls{equilibrium feasible} root configurations.
Thanks to the computationally efficient verification of the \gls{reachc}, we are able to solve \Pa much faster than previous approaches.

%~ Our contribution to \Pa is a generic characterization of the properties that the guide path must satisfy, in particular to enforce the completeness of the acyclic contact planner. 
Our contribution to \Pb is a fast contact generation scheme that can take into
account criteria to optimize user-defined properties.

Our results demonstrate that our method allows a pragmatic compromise between three 
criteria that are hard to conciliate: generality, performance, and quality of the solution, making it the first acyclic contact
planner compatible with \gls{interactive} applications.
%
\textbf{Regarding generality}, the \gls{reachc}, coupled with an approach based on limb decomposition, 
allows the method to address arbitrary multiped robots in \gls{cluttered} problems. The only pre-requisite is the specification 
of the volumes $W^0$.
%
\textbf{Regarding performance}, our framework is really efficient in addressing both \Pa and $\mathcal{P}_2$. This results in \gls{interactive} computation times.
%
\textbf{Regarding the quality of the paths}, the \gls{reachc} allows us to compute
\gls{equilibrium feasible} paths in all the presented scenarios, with low rejection rates.
As for \cite{Bouyarmane2009}, failures can still occur, due to the approximate condition used to compute the guide path.
The low computational burden of our framework however allows for fast re-planning in case of failure.



One direction for future work is to focus on a more accurate formulation of $C_{reach}^0$ to address 
this limitation.
Our main objective is of course to generate the complete interpolation between the contact sequences.
We have already obtained some success for some of the computed sequences~\citep{Carpentier2016}, but additional
work is required on the planning side to obtain a seamless workflow. To achieve this, we are currently working on the notion of transition certificate, i.e. formulating
conditions that guarantee that the interpolation between two contact configurations is dynamically feasible.
%~ Our current direction is to propose a geometric quantification of both the kinematic and dynamic constraints of the problem, expressed
%~ at the center of mass of the robot.
 %~ We are also
%~ working on new heuristics more closely related to the dynamics of the systems (for instance, choose the contacts that minimize the jerk between two contact configurations).
Lastly, we aim at going beyond static equilibrium and performing kinodynamic planning. We believe that the most promising direction in this regard is to integrate
the notion of Admissible Velocity Propagation in our current work \citep{DBLP:conf/rss/PhamCN13}.
Addressing these two issues is essential to bridge the gap between the planning and control aspects of multiped locomotion.
