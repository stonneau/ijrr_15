 \section{Discussion and future work} 
\label{sec:conclusion}

In this paper we consider acyclic contact planning in cluttered environments, formulated as two sub problems that we address sequentially.
The first problem is \Pa, that consists in computing a guide path for the root of the robot that can be extended; the second problem is \Pb, that is the computation of a discrete sequence of contacts along the root path.
Our contribution to \Pa is a generic characterization of the properties that the guide path must satisfy, in particular to enforce the completeness of the acyclic contact planner. We introduced a low-dimensional space $C_{reach}$ that can be mapped 
into the contact sub-manifold of the robot, approximated, and efficiently sampled by our Reachability-Based planner.
Our contribution to \Pb is a pragmatic contact generation scheme that can take into
account criteria to enforce interesting properties on the generated contacts (such as robustness, energy efficiency or naturalness).

Aside from the theoretical contributions, our results demonstrate that our method allows a very interesting compromise between three 
criteria that are hard to conciliate: generality, performance, and quality of the solution, making it the first acyclic contact
planner compatible with \gls{interactive} applications.
%
\textbf{Regarding generality}, the \gls{reachc}, coupled with an approach based on limb decomposition, 
allows the method to address arbitrary multiped robots. The only pre-requisite is the specification 
of the volumes $W^0$.
%
\textbf{Regarding performance}, our framework is really efficient in addressing both \Pa and \Pb. This results in computation costs close to real-time in
known static environments.
%
\textbf{Regarding the quality of the paths}, a parametrization of the \gls{reachc} allows us to compute
relevant paths in all the presented scenarios, with low rejection rates.
As for \cite{Bouyarmane2009}, failures can still occur, due to the compromise criterion used in computing the guide path.
The low computational burden of our framework however allows for fast re-planning in case of failure.

One direction for future work is to focus on a more accurate formulation of $C_{reach}$ to address 
this limitation.
Our main objective is of course to generate the complete interpolation between the contact sequences.
We have already obtained some successes for some of the computed sequences~\citep{Carpentier2016}, but we admit that additional
work is required on the planning side to obtain a seamless workflow. To achieve this, we are currently working on the notion of transition certificate, i.e. formulating
conditions that guarantee that the interpolation between two contact configurations is dynamically feasible.
Our current direction is to propose a geometric quantification of both the kinematic and dynamic constraints of the problem, expressed
at the center of mass of the robot.
 We are also
working on new heuristics more closely related to the dynamics of the systems (for instance, choose the contacts that minimize the jerk between two contact configurations).
Lastly, we aim at going beyond static equilibrium and to perform kinodynamic planning in the configuration space. We believe that the most promising direction in this regard is to integrate
the notion of Admissible Velocity Propagation in our current work \citep{DBLP:conf/rss/PhamCN13}.
Addressing these three issues is essential to bridge the gap between the planning and control aspects of multiped locomotion.
