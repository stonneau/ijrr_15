 \section{Discussion} 
\label{sec:discussion}

\subsection{validity and interest of the approximations}

Again, we assume that, for \gls{cluttered} problems, most of the times \contactreachability\, implies \equilibriumfeasibility.
Under this assumption, our goal is to find a \gls{contact reachable} guide path.

To generate such a path efficiently, ideally we need to only sample \gls{contact reachable} root configurations.
This requires exhibiting a necessary and sufficient condition for \contactreachability. %,that does not need the computation of the whole-body configuration.
By default, verifying \contactreachability\, implies a constructive demonstration by exhibiting a valid $\mathbf{q}^{\overline{0}}$. This is the approach chosen by \cite{Bouyarmane2009},
which is too computationally expensive.
To formulate a cheaper condition, we may turn our attention towards either an only-necessary or an only-sufficient condition.

Only-necessary conditions are appealing because they preserve the completeness of the search, while reducing 
the search space: they provide an outer approximation of $C_{Contact}^0$.
On the other hand, only-sufficient conditions provide the guarantee that any configuration that satisfies them is indeed \gls{contact reachable}:
they provide an inner approximation of $C_{Contact}^0$.

In practice, the only-necessary and only-sufficient conditions that we can provide are trivial and give rather inaccurate approximations of $C_{Contact}^0$.
Therefore, we propose a compromise between them: the \textit{reachability condition}, which is computationally efficient
and provides a rather accurate approximation of $C_{Contact}^0$.

TODO equilibirum as contact reachability

\begin{figure}[t]
\centering
  \begin{overpic}[width=1\linewidth]{figures/2D_feas}
		\put (15,) {$C_{Equil}^0$      $\subset$} 
		\put (47,) {$C_{Contact}^0$ $\approx$ } 
		\put (76,) {$C_{Reach}^0$} 
	\end{overpic}
\caption{Illustration of several root configurations sets used in this paper in a 2D scene. Obstacles are violet, and units are in meters. To show the sets in a 2D representation, all the rotational joints of HRP-2 are locked in the shown configuration, such that a torso configuration
is only described by two positional parameters (x and y). The root of the robot is indicated with a black cross. To compute the reachable workspace, the point on the ankle indicated by a green cross was used. $C_{Equil}^0$ is included in $C_{Contact}^0$. $C_{Reach}^0$ approximates $C_{Contact}^0$. Depending on a parametrization, we can obtain $C_{Contact}^0 \subset C_{Reach}^0$. Considering the configurations around the top obstacle, we can observe a dramatic
divergence between  $C_{Equil}^0$  and $C_{Contact}^0$ when the problem is not \textit{\gls{cluttered}}.}
		   \label{fig:dedefeas}
\end{figure}

Figure~\ref{fig:dedefeas} gives an insight into the difference between the three conditions, depicting $C_{Equil}^0$, $C_{Contact}^0$ and $C_{Reach}^0$, which are the sets of \gls{equilibrium feasible}, \gls{contact reachable} and \textit{reachable} root configurations, respectively.


TODO blah regarding contact plan generation, and ignoring the mid contact conf
The first constraint is less restrictive than the one proposed by \citeauthor{DBLP:conf/iser/EscandeKMG08}, because our method
allows to break and create one contact 

%~ \subsection{feasibility }

\section{Conclusion} 
\label{sec:conclusion}

In this paper we consider the \gls{cluttered} contact planning problem, formulated as two sub-problems that we address sequentially.
The first problem $\mathcal{P}_1$ consists in computing an \gls{equilibrium feasible} guide path for the root of the robot;
 the second problem $\mathcal{P}_2$ is the computation of a discrete sequence of whole-body configurations along the root path.

Our contribution to \Pa is the introduction of a low-dimensional space $C_{reach}$, an approximation of the space of \gls{equilibrium feasible} root configurations.
Thanks to the computationally efficient verification of the \gls{reachc}, we are able to solve \Pa much faster than previous approaches.

%~ Our contribution to \Pa is a generic characterization of the properties that the guide path must satisfy, in particular to enforce the completeness of the acyclic contact planner. 
Our contribution to \Pb is a fast contact generation scheme that can take into
account criteria to optimize user-defined properties.

Our results demonstrate that our method allows a pragmatic compromise between three 
criteria that are hard to conciliate: generality, performance, and quality of the solution, making it the first acyclic contact
planner compatible with \gls{interactive} applications.
%
\textbf{Regarding generality}, the \gls{reachc}, coupled with an approach based on limb decomposition, 
allows the method to address arbitrary multiped robots in \gls{cluttered} problems. The only pre-requisite is the specification 
of the volumes $W^0$.
%
\textbf{Regarding performance}, our framework is really efficient in addressing both \Pa and $\mathcal{P}_2$. This results in \gls{interactive} computation times.
%
\textbf{Regarding the quality of the paths}, the \gls{reachc} allows us to compute
\gls{equilibrium feasible} paths in all the presented scenarios, with low rejection rates.
As for \cite{Bouyarmane2009}, failures can still occur, due to the approximate condition used to compute the guide path.
The low computational burden of our framework however allows for fast re-planning in case of failure.
Furthermore because of this approximation, the guide search is not complete. The choice is deliberate, because we are convinced
that it is necessary to trade completeness for efficiency at all stages of the planner.
However, one direction for future work is to focus on a more accurate formulation of $C_{reach}^0$ to improve
the approximation.

Another limitation of the method is that it currently only applies to \gls{cluttered} problems.
However, it should first be noted that many problems belong to this class: for instance all the problems
at the DRC were \gls{cluttered}. Our method is thus already useful for many cases.
One way of extending its range of application, that we 
consider for future work, is to include the equilibrium criterion when solving $\mathcal{P}_1$.
Considering the set of obstacles intersecting with the reachable workspace for a given root configuration as candidates surfaces, we can use them to verify the equilibrium criterion.
This would give us a necessary condition for \equilibriumfeasibility.


Regarding the interpolation between the contact sequences ($\mathcal{P}_3$), we have already obtained some success for some of the computed sequences~\citep{Carpentier2016}, but additional
work is required on the planning side to obtain a seamless workflow. To achieve this, we are currently working on the notion of transition certificate, i.e. formulating
conditions that guarantee that the interpolation between two contact configurations is dynamically feasible.
%~ Our current direction is to propose a geometric quantification of both the kinematic and dynamic constraints of the problem, expressed
%~ at the center of mass of the robot.
 %~ We are also
%~ working on new heuristics more closely related to the dynamics of the systems (for instance, choose the contacts that minimize the jerk between two contact configurations).
A last limitation of our method is that only static equilibrium configurations are considered for contact planning.
We aim at performing kinodynamic planning to overcome this limitation. We believe that the most promising direction in this regard is to integrate
the notion of Admissible Velocity Propagation in our current work \citep{DBLP:conf/rss/PhamCN13}.
Addressing these two last issues is essential to bridge the gap between the planning and control aspects of multiped locomotion.
