 \section{Discussion and future work} 
\label{sec:conclusion}

In this paper we consider acyclic contact planning in cluttered environments, formulated as two sub problems that we address sequentially:
\Pa: the computation of a guide trajectory for the root of the robot that can be extended ; \Pb: the computation
of a discrete sequence of contacts along this trajectory.
Our contribution to \Pa is a generic characterization of the properties that the guide trajectory must staisfy, in particular to enforce the completeness of the acyclic contact planner. We introduced a low dimensional space  $C_{reach}$ that can be mapped 
into the contact submanifold of the robot, approximated and efficiently sampled by our Reachability-Based planner.
Our contribution to \Pb is a pragmatic contact generation scheme that can take into
account criteria to enforce interesting properties on the generated contacts (such as robustness, energy efficiency or naturalness).

Aside from the theoretical contributions, our results demonstrate that our method allows a very interesting compromise between three 
criteria that are hard to conciliate: generality, performance, and quality of the solution, making it the first acyclic contact
planner compatible with interactive applications.
%
\textbf{Regarding generality}, the reachability condition, coupled with an approach based on limb decomposition, 
allows the method to address arbitrary multiped robots. The only pre-requisite is the specification 
of the volumes $W^0$.
%
\textbf{Regarding performance}, our framework outperforms existing methods in addressing either \Pa or \Pb, leading to computation costs close to real-time in
statically known environments.
%
\textbf{Regarding the quality of the trajectories}, a parametrization of the reachability condition allows us to compute
relevant trajectories in all the scenarios presented, with low rejection rates.
As for \cite{Bouyarmane2009}, failures can still occur, due to the compromise criterion used in computing the guide trajectory.
The low computational burden of our framework however allows for fast replanning when the solution fails.

One direction for future work is to focus on a more accurate formulation of $C_{reach}$ to address 
this limitation.
Our main objective is of course to generate the complete interpolation between the contact sequences.
We have already been able to achieve this for HRP-2 for some of the sequences computed \citep{Carpentier2016}, but we concede that additional
work is required on the planning side to obtain a seemless workflow. To achieve this, we are currently working on the notion of transition certificate, ie formulating
conditions that guarantee that the interpolation between two contact configurations is dynamically feasible. We are also
working on new heuristics more closely related to the dynamics of the systems (for instance, choose the contacts that minimize the jerk between two contact configurations).


