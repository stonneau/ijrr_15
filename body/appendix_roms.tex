%~ \clearpage
\section{Generating the $W$ volumes for HRP-2}
\label{app:rom}

We detail our method to generate the volumes $W$ used
in RB-RRT, with the example of HRP-2.
The kinematic tree is split into four limbs $R^k$.
The arms are connected to the shoulders, and the legs to the root.
The obtained volumes $W$ are shown in Figure~\ref{fig:hrp2_w}.

\begin{figure}
\centering
  \begin{overpic}[width=1\linewidth]{figures/hrp2_w}
	\end{overpic}
\caption{The $W$ volumes computed for HRP-2. The red shapes are $W^0$. The green shapes represent the $W^k$.}
		   \label{fig:hrp2_w}
\end{figure}

\subsection{Step 1: computing the reachable workspace $W^k$ of a limb}


To generate a volume $W^k$, we proceed as follows:
\begin{enumerate}
\item Generate randomly $N$ valid limb configurations for $R^k$, for $N$ really large (say $100000$);
\item For each configuration, store the 3D position of the end effector joint relatively to the root of $R^k$; then compute the convex hull of the resulting point cloud;
\item The resulting polytope can contain a very large number of faces. A last step is thus to simplify it in a conservative way with the blender decimate tool (\url{http://wiki.blender.org/index.php/Doc:2.4/Manual/Modifiers/Generate/Decimate}). For HRP-2 we apply the operator with a ratio of $0.06$, resulting in a polytope of 38 faces for the arms and the legs.
\end{enumerate}
  
\begin{figure}
\centering
  \begin{overpic}[width=1\linewidth]{figures/roms}
	\end{overpic}
\caption{Different approximations of the range of motion of the right arm of HRP-2. Left: non convex-hull, computed with the powercrust algorithm~\citep{Amenta:2001:PC:376957.376986}. Middle:
convex hull of the reachable workspace. Right: Simplified hull used in our experiments.}
		   \label{fig:hrp2_roms}
\end{figure}

Figure~\ref{fig:hrp2_roms} illustrate the obtained $W^k$ for HRP-2.
Regarding the procedure, we can see that step 2 is conservative (Figure~\ref{fig:hrp2_roms}--right), which 
is acceptable, especially because the lost set essentially relates to configurations close to singularity (they are close to the boundaries of the reachable workspace, and
often not \gls{contact reachable}, as illustrated in Figure~\ref{fig:dedefeas}, where the exterior boundaries of the reachable workspace appear
red, thus not belonging to $C_{Contact}^0$). We choose again to be less complete but more efficient, regarding the number of collision tests to be performed by RB-RRT.
In step 1 on the other hand, selecting the convex hull (Figure~\ref{fig:hrp2_roms}--middle) instead of a minimum encompassing shape (Figure~\ref{fig:hrp2_roms}--left) may introduce false positives.
Concretely, because the false positive set intersects with $W^0$, the scaling volume of the robot torso, the induced error is compensated,
as verified by the results shown by Table~\ref{tab:requestpercent}.

\subsection{Step 2: computing the torso scaling workspace $W^0$ of the robot}
To define the volume $W^0$ of HRP-2, we proceed in an empirical manner.
First, we compute the bounding boxes of the robot torso, head, and upper legs (Figure~\ref{fig:hrp2_w} -- red shapes).
Then, we perform a scaling of these boxes by a factor $s$. 
The higher $s$ is, the more likely sampled configurations are to be feasible, but the less complete is the approach.
To compute the appropriate value of $s$, we proceed as described in Section~\ref{sec:params}, and choose empirically
$s^*=1.2$ as the appropriate value for HRP-2.
