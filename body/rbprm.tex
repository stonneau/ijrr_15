% !TEX root =  ../main.tex
%~ \section{Computing a guide path in $C^0_{reach}$ ($\mathcal{P}_1$) }
\section{Root path planning in the contact reachable space}
\label{rbprm}

During the root path planning we only consider the root configuration $\mathbf{q}^0$ defined in the previous Section,
as well as the environment $O$.

Given a start and a goal configurations, we aim at computing a guide path $\mathbf{q}^0(t) : [0,1] \longrightarrow$ $\mathbb{R}^r$ verifying:
\begin{equation*} \label{eq:path}
\forall t \in [0,1], \mathbf{q}^0(t)  \in C_{Equil}^0
\end{equation*}
This means that any root configuration must be extended into a whole-body, static equilibrium configuration.
$C_{Equil}^0$  cannot be described analytically.

The main hypothesis of this work is that for a large variety of locomotion tasks, we can define a space  $C_{Reach}^0 \simeq C_{Contact}^0$, such that 
\begin{equation} \label{eq:creach}
\forall t \in [0,1], \mathbf{q}^0(t) \in C_{Reach}^0 \Rightarrow \mathbf{q}^0(t)  \in C_{Equil}^0
\end{equation}
% for \gls{cluttered} problems.
We call  $C_{Reach}^0$ the \textit{contact reachable workspace}, and detail its construction in the following.
The validity of this hypothesis is discussed in depth in Section~\ref{sec:discussion}.
%~ We describe the construction of $C_{Reach}^0$ in the following.
 
\subsection{Conditions for contact reachability}
The contact reachable workspace is defined as a compromise between two necessary and a sufficient condition for contact creation.

\textbf{necessary conditions:}
For a contact to be possible, an obstacle $O_i \subset O$ necessarily intersects the reachable workspace $W(\mathbf{q}^{0})$ of the robot (Figure~\ref{fig:contact_gen}--1). Also the torso of the robot $W^0(\mathbf{q}^{0})$ must necessarily be collision-free.
Therefore we can define an outer approximation  $ \mathcal{C}^0_{\textrm{\it Nec}} \supset$ $C_{Contact}^0$ as: 
\begin{equation}
\mathcal{C}^0_{\textrm{\it Nec}} = \{ \mathbf{q}^0 : W(\mathbf{q}^{0}) \cap O \neq \emptyset \textrm{ \textbf{and}}\ W^0(\mathbf{q}^{0}) \cap O = \emptyset \} % \\ }
 %~ & \text{ and } & A_{torso}(\mathbf{q}^{root}) \cap W = \emptyset \}
\end{equation}
%~ The inclusion $C_{Contact}^0$ $\subset \mathcal{C}^0_{\textrm{\it Nec}}$ is straightforward, but it is very important.
%~ Planning in $\mathcal{C}^0_{\textrm{\it Nec}}$ allows a strong reduction of the search space, while not discarding 
%~ any root configuration that could lead to a solution.
 
%~ The condition defining $\mathcal{C}^0_{\textrm{\it Nec}}$ is only necessary. This means that it might not be possible to extend a guide path planned 
%~ in $\mathcal{C}^0_{\textrm{\it Nec}}$ into a sequence of contact configurations.

\textbf{sufficient condition:}
Similarly we can define an inner approximation \mbox{$\mathcal{C}^0_{\textrm{\it Suf}} \subset $ $C_{Contact}^0$} by considering a bounding volume $B^{\textrm{\it Suf}}$ encompassing the whole robot in a given pose, except for the effector surfaces. 
\begin{equation}
\mathcal{C}^0_{\textrm{\it Suf}} = \{ \mathbf{q}^0 : W(\mathbf{q}^{0}) \cap O \neq \emptyset \textrm{ \textbf{and}}\ B^{\textrm{\it Suf}}(\mathbf{q}^{0}) \cap O = \emptyset \} % \\ }
 %~ & \text{ and } & A_{torso}(\mathbf{q}^{root}) \cap W = \emptyset \}
\end{equation}

%~ In general, the inclusion is strict, which means that we lose the completeness of the two-stage contact planner (i.e. the planner is not able to discover a path inside \mbox{$C_{reach} \setminus \mathcal{C}^0_{\textrm{\it Suf}}$}). However, the sufficient condition guarantees that any such path leads to a valid sequence of contacts.

\subsection{The compromising reachability condition}
\label{sec:scaling}
The ideal shape $B^*, W^0 \subset B^* \subset B^{\textrm{\it Suf}}$ would define a necessary \textbf{and} sufficient condition for contact creation. 
It would guarantee that any root configuration $\mathbf{q}^{0} \in B^*$ would result in a contact configuration, while any $\mathbf{q}^{0} \notin B^*$ could not.
To our knowledge  $B^*$ has no explicit definition.
Therefore, we approximate $B^*$ to define the contact reachable space $C_{Reach}^0$.

%~ However, using a shape between $W^0$ and $B^{\textrm{\it Suf}}$ leads to a trade-off between a necessary and a sufficient condition. 
We define $W^0_s$ as the volume $W^0$ subject to a scaling transformation by a factor $s \in \mathbb{R}^+$.
%
We then consider the spaces $C_{s}^0$
 \begin{equation}
 \label{eq:reachc}
C^0_s = \{ \mathbf{q}^0 : W(\mathbf{q}^{0}) \cap O \neq \emptyset \textrm{ \textbf{and}}\ W^0_s(\mathbf{q}^{0}) \cap O = \emptyset \} % \\ }
 %~ & \text{ and } & A_{torso}(\mathbf{q}^{root}) \cap W = \emptyset \}
\end{equation}
%
The parametrization of $s$ defines a trade-off:
If $s=1$, then $W^0_s$ = $W^0$, such that $C_1^0$ = $\mathcal{C}^0_{\textrm{\it Nec}}$.
 %~ We thus consider that $s \geq 1$, since smaller values would only worsen the approximation.
By increasing $s$, the condition can become sufficient, but less and less necessary.  
Eq.~\ref{eq:reachc} thus defines the \textit{reachability condition}. We fix a value $s^*$ for $s$ and define  $C_{Reach}^0 = C^0_{s^*}$.
The computation of $s^*$ is detailed in Section~\ref{sec:params}. 
In Appendix~\ref{app:rom}, we give a generic method to compute the $W$ volumes appearing in the definition of $C_{Reach}^0$.

\subsection{Computing the guide path in $C_{reach}^0$}
$C_{Reach}^0$ can be sampled efficiently thanks to Eq.~\ref{eq:reachc}, and can thus be used with any standard motion planner.
%~ The only significant change is to replace the collision checking with the \textit{reachability condition}.
Our current implementation uses the Bi-RRT planner \citep{770022} provided by the HPP software~\citep{7759083}.
Our implementation is exactly the same as the pseudo-code of the original planner (which does not detail the configuration validation method). With respect to a ``classic'' implementation, the only difference is that instead of validating a configuration using collision detection, we validate it with the \textit{reachability condition}.\\

%~ \delst{However, to improve the sampling efficiency 
%~ we bias the sampling process to generate near-obstacle configurations, similarly to cite{Amato98choosinggood}.}

This Section has presented a guide path planner for the geometric root of a robot, implemented as a low-dimensional sampling-based 
algorithm. Given start and goal configurations, it outputs a continuous path for the robot's root. 
%~ \deladp{Thanks to the \textit{reachability condition}, we assume that for any configuration in the path there exists a joint configuration that results in static equilibrium.}
%~ Thanks to these modifications, the problem of planning a \gls{contact reachable} path is reduced to a geometric collision-checking problem, of low dimension (6 for HyQ, 8 for HRP-2 that has 2 joints in the torso).
%~ By doing this we can solve the problem with a sampling-based approach, in \gls{interactive} computation times.
