% !TEX root =  ../main.tex
\section{Computing a guide path in $C^0_{reach}$ ($\mathcal{P}_1$) }
\label{rbprm}

We consider the issue of computing an \gls{equilibrium feasible} guide path $\mathbf{q}^0(t) : [0,1] \longrightarrow$ \gls{ccequil0} for the root of a multiped robot, connecting user-defined start and goal configurations.

Again, we assume that, for \gls{cluttered} problems, most of the times \contactreachability\, implies \equilibriumfeasibility.
Under this assumption, our goal is to find a \gls{contact reachable} guide path.

To generate such a path efficiently, ideally we need to only sample \gls{contact reachable} root configurations.
This requires exhibiting a necessary and sufficient condition for \contactreachability. %,that does not need the computation of the whole-body configuration.
By default, verifying \contactreachability\, implies a constructive demonstration by exhibiting a valid $\mathbf{q}^{\overline{0}}$. This is the approach chosen by \cite{Bouyarmane2009},
which is too computationally expensive.
To formulate a cheaper condition, we may turn our attention towards either an only-necessary or an only-sufficient condition.

Only-necessary conditions are appealing because they preserve the completeness of the search, while reducing 
the search space: they provide an outer approximation of \gls{ccontact0}.
On the other hand, only-sufficient conditions provide the guarantee that any configuration that satisfies them is indeed \gls{contact reachable}:
they provide an inner approximation of \gls{ccontact0}.

In practice, the only-necessary and only-sufficient conditions that we can provide are trivial and give rather inaccurate approximations of \gls{ccontact0}.
Therefore, we propose a compromise between them: the \textit{reachability condition}, which is computationally efficient
and provides a rather accurate approximation of \gls{ccontact0}. % for \gls{cluttered} problems.

%~ As said in the previous section, the goal is to enforce that the guide can be discretized into a sequence of \gls{equilibrium feasible} configurations. We denote by \gls{ccequil} the equilibrium sub-manifold of the robot.

%~ We say that a root configuration $\mathbf{q}^{0}$ is \gls{equilibrium feasible} if it belongs to the set:
%~ \begin{equation}
      %~ C_{Equil}^0= \{ \mathbf{q}^{0}: \exists \mathbf{q}^{\overline{0}}, \mathbf{q}^{0} \oplus \mathbf{q}^{\overline{0}} \in C_{Equil} \}
    %~ \label{eq:pi}
%~ \end{equation}

%~ The set of all \textit{geometrically feasible} root configurations is denoted by $C_{reach}$.
%~ By extension, a path $\mathbf{q}^0(t)$ is \gls{equilibrium feasible} if $\forall t \in [0,1], \mathbf{q}^0(t) \in C_{contact}^0$.


%~ In this Section we characterize \gls{ccontact0} theoretically with a necessary and a sufficient condition for belonging to the set.
%~ We then approximate \gls{ccontact0} as a space $C_{reach}$, where we can efficiently sample configurations.
%~ We finally describe our implementation of an efficient RRT planner in $C_{reach}$.

   
%~ We want all the configurations of $q^0(t)$ to belong to a set $C_{reach}$ defined as:
%~ 
%~ \begin{eqnarray}
%~ C_{reach}= \{ \mathbf{q}^{0} & : & \exists f, f(\mathbf{q}^{0}) \in C_{contact} \}
%~ \end{eqnarray}
 
\subsection{Conditions for contact reachability}
%~ By default, \glslink{contact reachable}{\textit{contact reachability}} implies a constructive demonstration by exhibiting a valid $\mathbf{q}^{\overline{0}}$. This is the approach chosen by \cite{Bouyarmane2009}. However, computing $\mathbf{q}^{\overline{0}}$ is expensive in terms of computation time. In this section we rather define a necessary condition and a sufficient condition for \glslink{contact reachable}{\textit{contact reachability}} that do not require this explicit computation.




\subsubsection*{contact reachability, a necessary condition:}
For a contact to be possible, a volume $O_i \in O$ necessarily intersects with the reachable workspace $W(\mathbf{q}^{0})$ (Figure~\ref{fig:contact_gen}--1). Furthermore, if $\mathbf{q}^{0}$ is \gls{contact reachable}, then the torso of the robot $W^0(\mathbf{q}^{0})$ is necessarily not colliding  with the environment $O$.

Therefore we can define an outer approximation  $ \mathcal{C}^0_{\textrm{\it Nec}} \supset$ \gls{ccontact0} defined as: 
\begin{equation}
\mathcal{C}^0_{\textrm{\it Nec}} = \{ \mathbf{q}^0 : W(\mathbf{q}^{0}) \cap O \neq \emptyset \text{ and } W^0(\mathbf{q}^{0}) \cap O = \emptyset \} % \\ }
 %~ & \text{ and } & A_{torso}(\mathbf{q}^{root}) \cap W = \emptyset \}
\end{equation}
%~ The inclusion \gls{ccontact0} $\subset \mathcal{C}^0_{\textrm{\it Nec}}$ is straightforward, but it is very important.
%~ Planning in $\mathcal{C}^0_{\textrm{\it Nec}}$ allows a strong reduction of the search space, while not discarding 
%~ any root configuration that could lead to a solution.
 
%~ The condition defining $\mathcal{C}^0_{\textrm{\it Nec}}$ is only necessary. This means that it might not be possible to extend a guide path planned 
%~ in $\mathcal{C}^0_{\textrm{\it Nec}}$ into a sequence of contact configurations.

\subsubsection*{contact reachability, a sufficient condition:}
%~ In building $C_{reach}^1$, we directly considered the including hull of the root body $W^{0}$, obtaining a necessary condition. 
A trivial sufficient condition for \glslink{contact reachable}{\textit{contact reachability}} can be constructed as a variation of $\mathcal{C}^0_{\textrm{\it Nec}}$, by replacing $W^0$ with a bounding volume $B^{\textrm{\it Suf}}$ encompassing the whole robot in a given pose, except for the effector surfaces to be in contact. We denote by \mbox{$\mathcal{C}^0_{\textrm{\it Suf}} \subset $ \gls{ccontact0}} the set of root configurations corresponding to this sufficient condition:

\begin{equation}
\mathcal{C}^0_{\textrm{\it Suf}} = \{ \mathbf{q}^0 : W(\mathbf{q}^{0}) \cap O \neq \emptyset \text{ and } B^{\textrm{\it Suf}}(\mathbf{q}^{0}) \cap O = \emptyset \} % \\ }
 %~ & \text{ and } & A_{torso}(\mathbf{q}^{root}) \cap W = \emptyset \}
\end{equation}

%~ In general, the inclusion is strict, which means that we lose the completeness of the two-stage contact planner (i.e. the planner is not able to discover a path inside \mbox{$C_{reach} \setminus \mathcal{C}^0_{\textrm{\it Suf}}$}). However, the sufficient condition guarantees that any such path leads to a valid sequence of contacts.

\subsection{contact reachability: a compromise \gls{reachc}}
The sufficient condition is not interesting in practice since it leads the solver to lose too many interesting paths. The necessary condition is not perfect either, since the first stage of the planner would stop on a guide that is not \gls{contact reachable}. 
An ideal shape $B$ (with $W^0 \subset B \subset B^{\textrm{\it Suf}}$) that leads to a necessary and sufficient condition may exist---even if it seems intuitively very unlikely in general. 
%~ Its construction is out of the scope of this work.

%~ It might be possible to find an ideal shape $B$ that leads to a necessary and sufficient condition; however, it seems intuitively very unlikely in general. The construction of a shape $W^0 \subset B \subset B^{\textrm{\it Suf}}$ leading to a necessary and sufficient condition (or the proof of its nonexistence) is out of the scope of this work.

However, using a shape between $W^0$ and $B^{\textrm{\it Suf}}$ leads to a trade-off between a necessary and a sufficient condition. We define $W^0_s$ as the volume $W^0$ subject to a scaling transformation by a factor $s \in \mathbb{R}^+$.
%
We then consider the spaces $C_{s}^0$
 \begin{equation}
C^0_s = \{ \mathbf{q}^0 : W(\mathbf{q}^{0}) \cap O \neq \emptyset \text{ and } W^0_s(\mathbf{q}^{0}) \cap O = \emptyset \} % \\ }
 %~ & \text{ and } & A_{torso}(\mathbf{q}^{root}) \cap W = \emptyset \}
\end{equation}
%
If $s=1$, then $W^0_s$ = $W^0$, such that $C_1^0$ = $\mathcal{C}^0_{\textrm{\it Nec}}$. We thus consider that $s \geq 1$, since smaller values would only worsen the approximation.
By increasing $s$, the condition can become sufficient.
%~ The higher $s$, the closer this \gls{reachc} is to being sufficient, and if $s=1$, the search is complete (in the sense that all \gls{contact reachable} configurations can be found). 
The parametrization of $s$ then defines a trade-off between these two interesting extremes. 
We can choose $s$ by hand, or automatically as explained in Section~\ref{sec:params}.
The chosen value $s=s^*$ defines the \gls{reachc}, therefore we write \gls{creach0} $= C^0_{s^*}$.
%~ We call this condition the \gls{reachc}, which is defined by the chosen value $s=s^*$, and define \gls{creach0} $= C^0_{s^*}$.
%~ In the rest of the paper, we consider that an appropriate value $s^*$ of $s$ is chosen, and define \gls{creach0} $= C^0_{s^*}$.

%~ As an example, Figure~\ref{fig:HyQ_roms} presents the volumes $W$ computed for the HyQ robot.
In Appendix~\ref{app:rom}, we give a generic method to compute the volumes appearing in the definition of \gls{creach0}, with the example of HRP-2.
%~ Section~\ref{sec:results} shows that in practice, it is easy to adjust $s$ to keep most of the interesting guides without introducing incorrect guides.

\subsection{Computing the guide path in $C_{reach}^0$}
Any sampling-based motion planner can be used to plan a path in \gls{creach0}. 
Indeed, contrary to \gls{ccontact}, \gls{creach0} has a nonzero measure in the configuration space $C$. Therefore a standard uniform sampling approach
can work, in spite of a high rejection rate. 
Thus, the only significant change regarding a classical planner is to replace the collision checking with the \gls{reachc} when verifying
the drawn configurations and associated local paths.

However, to improve the sampling efficiency 
%~ while remaining probabilistically complete, 
we bias the sampling process to generate near-obstacle configurations, similarly to~\cite{Amato98choosinggood}.
%~ First, a configuration is set to a random point on the surface of one randomly chosen obstacle. The root location is then translated and rotated randomly until the \gls{reachc} is satisfied.
Our current implementation of these modifications is based on the Bi-RRT planner \citep{770022} provided by the HPP software.

Thanks to these modifications, the problem of planning a \gls{contact reachable} path is reduced to a geometric collision-checking problem, of low dimension (6 for HyQ, 8 for HRP-2 that has 2 joints in the torso). By doing this we can solve the problem with a sampling-based approach, in \gls{interactive} computation times.
