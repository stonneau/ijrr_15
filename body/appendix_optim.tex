\section{Addressing $\mathcal{P}_3$}
\label{app:optim}
The stair climbing and the standing up scenarios were validated with the trajectory optimization scheme provided in \citeauthor{Carpentier2016}. 
To address the others scenarios, we propose a new implementation, entirely open source (\url{https://github.com/stonneau/python_sandbox}), which can be integrated directly in our motion planner software.
This implementation is slower, but handles collision avoidance in the trajectory generation phase.
It uses a modified formulation of the optimization problem, that we present here.

We rewrite Eq.~\ref{eq:new_eul} in the general case:

%~ \begin{align} \label{eq:new_eul}
%~ \underbrace{
%~ \mat{\mathbf{I}_3 & \dots & \mathbf{I}_3 \\
%~ \hat{\mathbf{p}}_1 & \dots & \hat{\mathbf{p}}_e} \mathbf{V}
%~ }_\mathbf{G} \bm{\beta}, = 
%~ \underbrace{\mat{\mathbf{0}_{3\times 3} \\ m \hat{\mathbf{g}}}}_{\mathbf{D}} \mathbf{c} + 
%~ \underbrace{\mat{-m\mathbf{g} \\ \mathbf{0}}}_{\mathbf{d}}
%~ \end{align}

\begin{align} \label{eq:new_eul_acc}
\mathbf{G} \bm{\beta} = 
\underbrace{\mat{m (\mathbf{\ddot{c}} - {g}) \\m (\mathbf{\ddot{c}} - {g}) + \dot{L}}}_{\mathbf{w}}
%~ \underbrace{m (\mathbf{\ddot{c}} - \mathbf{g})  \\ m (\mathbf{\ddot{dc}} - \mathbf{g}) + \mathbf{\dot{L}} }_{\mathbf{w}}
\end{align}

where $\dot{L}$ is the angular momentum expressed at the com.
Eq.~\ref{eq:new_eul_acc} defines a 6 dimensional cone $\mathcal{K}$ ~\citep{qiu:dhm:2011,Caron2015}. For a given set of contacts,
this cone determines all the admissible wrenches $\mathbf{w}$ such that the robot won't fall. The objective 
is thus to plan a trajectory for $c$ and $\dot{L}$ that remains inside $\mathcal{K}$. As a cone,
$\mathcal{K}$ can be described as a linear inequality, obtained with the double description method~\citep{Fukuda1996}:

\begin{equation}
\label{eq:cone_k}
	\mathcal{K} =  \left\{ \mathbf{w}, \mathbf{A}\mathbf{w} \leq \mathbf{b	} \right\}
\end{equation}

We now consider a two contact configurations  $\mathbf{q}_0$ and $\mathbf{q}_1$ given by our planner: in the general case one contact is broken and one created to get from
$\mathbf{q}_0$ to $\mathbf{q}_1$. The motion is thus constrained by three cones $\mathcal{K}_s, s =0 \dots 2$. We manually define the duration of each contact phase, of total
time $T$, and formulate the following optimization problem:

\begin{equation} \label{eq:lin_prog_rob} \begin{aligned}
%~ \find \quad & \mx{R} \in \Rv{3\times T}, \mxdd{C} \in  \Rv{6\times T} \\
\minimize_{\mathbf{c}(t), \mathbf{w}(t)}  \quad & \sum_{s=0}^{2} \int_{t_s}^{t_s + \Delta t_s}  h(\mathbf{c}, \mathbf{w})  \\
\st \quad & \mathbf{\dot{c}} = f(\mathbf{c}, \mathbf{w}) \\
	\quad & \mathbf{A}_s \mathbf{w}(t)  \leq \mathbf{b}_s , \forall t \in  \left[t_s, t_s + \Delta t_s \right[ \\
	\quad & \mathbf{D}_s \mathbf{c}(t)  \leq \mathbf{d}_s , \forall t \in  \left[t_s, t_s + \Delta t_s \right[ \\
	\quad & \mathbf{c}(0)  = \mathbf{c}_{\mathbf{q}_0} \\
	\quad & \mathbf{c}(T)  = \mathbf{c}_{\mathbf{q}_1} \\
	\quad & \mathbf{c}(0)  = \mathbf{\dot{c}}(0)  = \mathbf{\ddot{c}}(0) = \mathbf{0}\\
	\quad & \mathbf{c}(T)  = \mathbf{\dot{c}}(T)  = \mathbf{\ddot{c}}(T) = \mathbf{0}\\
    %~ \quad & ({\vct{r}}\tra\vct{r}=1,  \forall t  ?) \\
    %~ \quad & \delta t \sum\limits_1^{T-1}{\vc{r}^1}\tra {\vc{r}^1} \ge t_{transit} \\
    %~ \quad & \mx{A}_{\int} \mxdd{C} \leq \vc{a}_{\int} \\
    %~ \quad & \mx{V}_{\int} \mxd{C} \leq \vc{v}_{\int} \\
    %~ \quad & \mx{B}_{\int}^{1} \mx{C} \leq \vc{b}_{\int} \\
    %~ \quad & [\vc{c}_0\tra, \vcd{c}_0\tra]\tra = \vc{x}_0 \\
    %~ \quad & \vc{c}_T = \vc{c}_T^{input} \\
\end{aligned} \end{equation}

$h$ is a cost function (We typically minimize the angular momentum and acceleration over the whole trajectory, and use
a log barrier to constrain kinematically the center of mass position with respect to the contact locations). $f$ is the colocation method
bounding the position, velocity and acceleration of the center of mass. 
$\mathbf{D}_s$ and $\mathbf{d}_s$ denote stacked kinematic constraints on the center of mass position, determined by the active contact locations.
The inequalities for each contact are determined in the same way that the reachable workspace is computed in Appendix~\ref{app:rom}, with
the effector serving as root (a triangulated object is a polytope, and as such can be described as a set of linear inequalities).

This formulation can trivially be extended over the whole contact sequence.
In our implementation, the problem is discretized using time steps of 100 ms. 

As an output, the solution of this optimization problem gives us a admissible center of mass trajectory.
To compute the whole body motion, we use a two step approach.

First, we plan a kinematic motion for the robot, subject to the contact constraints and center of mass constraints.
This is achieved using a RRT planner. Using a sampling based planner guarantees that the effector will find a whole-body motion that is collision free.

Planning the whole-body takes approximatively 1.5 seconds for a complete contact transition.

This motion is then validated in our dynamic simulator~\citep{DelPrete2015b}, using the whole-body trajectory as a postural task to an inverse dynamics
solver, that accounts for the torque constraints of the robot. This final result is shown in the videos.
