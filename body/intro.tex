We consider the problem of planning the acyclic sequence of contacts describing the motion of a multiped robot in a cluttered environment. Acyclic contact planning is a particular class of motion planning where every configuration of the resulting trajectory must be in contact with the environment in order to support the balance of the system.

Most multipedal locomotion focus on cyclic walking gaits \citep{Kajita03a}. But executing
this behaviour on cluttered environments is dangerous, if not impossible.
in an analysis of their participation to the darpa challenge, \citeauthor{atkensondarpa}
noted: ``Except for egress, no robots in the DRC Finals used
the  stair  railings  or  any  form  of  bracing.   Even  drunk  people  are  smart  enough  to  use  nearby  supports.
Full body locomotion (handholds,  bracing,  leaning against a wall or obstacles) should be easier than our
current high performance minimum contact locomotion approaches."

Indeed, the current approach to locomotion planning is to avoid obstacles as much as possible, instead of using them
to faciliate locomotion. The reason for this, as the authors state, is that ``More contacts make tasks
mechanically easier, but algorithmically more complicated for planning, and the transitions are difficult to
both plan and control[...].  We have seen very few robot planners that are  capable of  generating this  behavior."

The difficulty of addressing such a problem comes both in practice from the proximity to the obstacles (that tends to make the sampling of valid configuration tedious) and in theory from the foliation of the configuration space, where zero-measure manifolds intersect in a combinatorial manner \citep{simeon-manipulation-04}.

\subsection{State of the art}

\newcommand{\Pa}{$\mathcal{P}_1$ }
\newcommand{\Pb}{$\mathcal{P}_2$ }

Additionally to robotics, acyclic motion planning is also a problem of interest in neurosciences and biomechanics, and virtual character animation.
Early contributions in the latter field rely on local adaptation of motion graphs \citep{citeulike:220163}, or ad-hoc construction of locomotion controllers \citep{Pettre:2003:LPD:846276.846313}. These approaches can intrinsically not adapt to new situations or discover complex behaviors in unforeseen contexts.

The issue of planning acyclic contacts was first completely described by \citeauthor{conf/iser/BretlRLKA04}, where  it is proven to require the handling of two simultaneous problems, $\mathcal{P}_1$: a relevant guide trajectory for the root of the robot in $SE(3)$; and $\mathcal{P}_2$: the planning of a discrete sequence of acyclic, balanced contact configurations along the trajectory\endnote{A third non trivial problem, $\mathcal{P}_3$, not adressed in this work, then consists in interpolating a complete motion between two postures of the contact sequence. Noticebly, \citeauthor{Carpentier2016} combine
our planner with such a method to generate the complete motion.}.  A key issue is to avoid combinatorial explosion when considering at the same time the possible contact configurations and the potential trajectories. This seminal paper proposes a first effective algorithm, able to handle simple situations (such as climbing scenarios), but not scalable to arbitrary environments. Following it, seve\-ral papers have applied this approach in particular situations, typically limiting the combinatorial by imposing a fixed set of possible contacts \citep{Hauser06usingmotion, stilman2010}.

Most of the papers that followed the work of \citeauthor{conf/iser/BretlRLKA04} have explored alternative formulations to handle the combinatorial issue. Two main directions have been explored. \textbf{On one hand, local optimization of both the root trajectory \Pa and the contact positions $\mathcal{P}_2$} has been used, to trade the combinatorial of the complete problem for a differential complexity, at the cost of local convergence. A complete example of the potential offered by such approaches is given by \cite{Mordatch:2012:DCB:2185520.2185539}, followed by a successful adaptation to a real robot \citep{mordatch2015}. To keep reasonable computation times, the method uses a simplified dynamic model for the avatar. Still, the computation time is far from interactive  (about 1 minute of computation for a sequence of 20 contacts). \cite{DBLP:conf/humanoids/DeitsT14} propose to solve contact planning globally as a mixed integer problem, but only cyclic, bipedal locomotion is considered. Aside from the computation cost, a major drawback of these optimization based approaches is thus that they only offer local convergence when applied to acyclic contact planning.

\textbf{On the other hand, the two problems \Pa and \Pb might be decoupled} to reduce the complexity. The feasibility and interest of the decoupling is shown by \citeauthor{DBLP:conf/iser/EscandeKMG08} who manually set up a rough guide trajectory (i.e. an ad-hoc solution to $\mathcal{P}_1$). \Pb  is then addressed as the combinatorial computation of a feasible contact sequence in the neighborhood of the guide. A solution can then be found, at the cost of prohibitive computation times (several hours). Furthermore, this approach is suboptimal because the quality of the motion is conditioned by the relevance of the guide trajectory,  which is not evaluated \textit{a priori}. \citeauthor{Bouyarmane2009} precisely focus on automatically computing a guide trajectory with guarantees of contact feasibility, by extending key frames of the trajectory into whole-body, balanced contact configurations. Randomly sampled configurations are projected into the contact submanifold using a generalized inverse kinematics solver, a computationally expensive process (about 15 minutes are required to compute a guide trajectory in the examples presented). Moreover this explicit projection is yet an insufficient condition and does not provide strong guarantees on the feasibility of the path between two key positions in the trajectory.
 

\subsection{Paper contribution and organization}
Although the optimization approach is promising, we choose to focus on the sample-based methodology, more able to find complex trajectories in cluttered environments. While the theoretical structure of the problem is well understood, there is currently no scalable method to solve it. The combinatorial of the original problem (as described by \citeauthor{conf/iser/BretlRLKA04}) is too high to have any hope of tackling it directly. Alternative formulations are necessary to obtain practical solutions. We believe that the separation between the guide trajectory and the contact sequence is the most promising direction \citep{DBLP:conf/iser/EscandeKMG08}. However, this direction raises two theoretical questions that remain to be solved, or even to be properly formulated:
\begin{itemize}
\item The guide trajectory must satisfy a property guaranteeing the existence of a contact sequence to actuate it\endnote{This property is related to the controllability of the root actuated by the contact forces, but for discrete bounded properties.}. This property has not been studied yet: the only way to validate a trajectory is to explicitly compute the contacts, which is computationally not reasonable \citep{Bouyarmane2009}.
\item There is an infinite combination of possible contact sequences for a given root trajectory. The selection of one particular contact sequence with interesting properties (minimum number of contact change, robustness, efficiency or naturalness) has been studied for cyclic cases \citep{Hauser06usingmotion}, but has not been efficiently applied to cluttered environments (\citeauthor{bouyarmane:lirmm-00777727, DBLP:conf/iser/EscandeKMG08} mostly randomly pick one contact sequence, leading to possibly very tedious contact sequences).

\end{itemize}
\begin{figure*}[t]
  \centering
  \begin{overpic}[width=1\linewidth]{figures/workflow_new}
    \put (6.4,1.8) {\normalsize{Request}} 
    \put (22,11.3) {\large{A} }
    \put (31.6,2) {\scriptsize{\color{white}RB-PRM}} 
    \put (46.4,11.3) {\large{B} }
    \put (72.5,2) {\tiny{\color{white}EFORT}} 
  \end{overpic}
  \vspace{-1em}
  \caption{
    Overview of our 2-stage framework. (A) Given a path request between the yellow and blue positions, a guide trajectory is computed in $C_{reach}$ using RB-PRM. (B) The trajectory is extended into a discrete sequence of contact configurations using EFORT.}
  \label{fig:framework}
\end{figure*}

We claim that the desirable contact properties of a guide trajectory, proposed by \citeauthor{Bouyarmane2009}, can be formulated in a space of lower dimension, which we call $C_{reach}$. This formulation can make the planning of a guide trajectory more efficient computationally, while providing equivalent guarantees to planning directly in the configuration space. Among the particular properties obtained when planning in $C_{reach}$, we would like to guarantee that any reduced trajectory can actually lead to a feasible sequence of contacts, in which case we say that the reduced trajectory is truly feasible. It is possible in theory to guarantee that any reduced trajectory is truly feasible, even if it is more efficient in practice to approximate this property. The true-feasibility of the guide trajectory then allows us to focus on the selection of one particular sequence of contacts, for example one that minimizes the number of contacts in the sequence or maximizes the robot efficiency or style.

Based on these fundamental observations, we implement a very efficient acyclic contact planner. Our method is based on a probabilistic roadmap (PRM), that computes offline guide trajectories that are approximately truly feasible. The planner then resolves online the contact sequence by refining a guide trajectory computed from the PRM. Our planner is able to compute physically-consistent contact sequences for very complex systems (a humanoid, 28 joints; and an insectoid, 48 joints) in a few seconds for classical scenarios like climbing, and less than a minute for very complex problems like egress from a damaged truck. The planner also generalizes to planning dexterous manipulation movements, as demonstrated by preliminary results.

The contributions of the paper are twofold. We propose the first theoretical characterization of today's most efficient practical approach to sampled-based planning of acyclic contacts. And based on this characterization, we propose a very efficient and general implementation of an acyclic contact planner, the first one compatible with interactive applications. 

We propose a framework to address the motion planning problem for multiped robots in cluttered environments: given a start and a goal configuration, the objective is to compute a sequence of contact configurations allowing to achieve the motion. For instance, we can consider the task of standing up, illustrated in Figure~\ref{fig:framework}--right. The problem is decoupled into two sequential phases: 1) the computation of a guide trajectory for the root of the robot; 2) the computation of a discrete sequence of contact configurations allowing to achieve the motion along the trajectory.
%
The remainder of this section presents the general organization of our method in Section~2. The two following sections 2 and 3 present respectively our answer to problems \Pa and \Pb. Finally, we propose a complete experimental validation of the planner with three very different kinematic chains (humanoid, insectoid and three-finger manipulator) in various scenarios.
