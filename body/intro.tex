% !TEX root =  ../main.tex
We consider the problem of planning an acyclic sequence of contacts describing the motion of a multiped robot in a cluttered environment. Acyclic contact planning is a particular class of motion planning where every configuration of the resulting trajectory must be in contact with the environment in order to maintain the equilibrium of the system.

Most multipedal locomotion systems focus on cyclic walking gaits \citep{Kajita03a}. However, executing
this behavior on cluttered environments is dangerous, if not impossible.
In an analysis of their participation to the Darpa Robotics Challenge, \citeauthor{atkensondarpa}
noted: ``Except for egress, no robots in the DRC Finals used
the  stair  railings  or  any  form  of  bracing.   Even  drunk  people  are  smart  enough  to  use  nearby  supports.
Full body locomotion (hand holds,  bracing,  leaning against a wall or obstacles) should be easier than our
current high performance minimum contact locomotion approaches."

Indeed, the current approach to locomotion planning is to avoid obstacles as much as possible, instead of using them
to facilitate locomotion. The reason for this, as the authors state, is that ``More contacts make tasks
mechanically easier, but algorithmically more complicated for planning, and the transitions are difficult to
both plan and control[...].  We have seen very few robot planners that are  capable of  generating this  behavior."
The difficulty of addressing such a problem comes both in practice from the proximity to the obstacles (that tends to make the sampling of collision-free configuration tedious) and in theory from the foliation of the configuration space, where zero-measure manifolds intersect in a combinatorial manner \citep{simeon-manipulation-04}.

Previous contributions that embrace the combinatorial provide complete approaches, not applicable in practice because they require hours of computations \citep{conf/iser/BretlRLKA04}.
More recent successes use a local solution, resulting in more reasonable computation times (still far from real time), at the cost of dynamically inconsistent behaviors \citep{Mordatch:2012:DCB:2185520.2185539}.
Neither global nor local methods present satisfying performances, because planning simultaneously the robot trajectory and the contacts that allow
its execution is too costly. 

As suggested by \citeauthor{Bouyarmane2009}, we believe that these two tasks can be treated sequentially within a global planner, while preserving the completeness of the search.
We go further and claim that this can be done at a much smaller cost, provided that necessary and sufficient conditions for contact feasibility can be formulated for the robot trajectory.
This paper presents a geometrical representation of these conditions, and a concrete implementation of such a decoupled planner.
This implementation results from a trade-off between a necessary and a sufficient condition for feasibility, allowing us to find solutions extremely fast.
while preserving a high success rate in the demonstrated scenarios.

In the remainder of this introduction, we discuss further the current state of the art. This allows us to situate more precisely our contributions.

\subsection{State of the art}

\newcommand{\Pa}{$\mathcal{P}_1$ }
\newcommand{\Pb}{$\mathcal{P}_2$ }

Additionally to robotics, acyclic motion planning is also a problem of interest in neurosciences, biomechanics, and virtual character animation.
Early contributions in the latter field rely on local adaptation of motion graphs \citep{citeulike:220163}, or ad-hoc construction of locomotion controllers \citep{Pettre:2003:LPD:846276.846313}. These approaches are intrinsically not able to adapt to new situations or discover complex behaviors in unforeseen contexts.

The issue of planning acyclic contacts was first completely described by \citeauthor{conf/iser/BretlRLKA04} in their seminal paper. The issue requires the simultaneous handling of two problems, $\mathcal{P}_1$: Planning a relevant guide trajectory for the root of the robot in $SE(3)$; and $\mathcal{P}_2$: Planning a discrete sequence of acyclic contact configurations along the trajectory (A third nontrivial problem, $\mathcal{P}_3$, not addressed in this work, then consists in interpolating a complete motion between two postures of the contact sequence).  A key issue is to avoid combinatorial explosion when considering at the same time the possible contact configurations and the potential trajectories. This seminal paper proposes a first effective algorithm, able to handle simple situations (such as climbing scenarios), but not applicable to arbitrary environments. Following it, seve\-ral papers have applied this approach in particular situations, typically limiting the combinatorial by imposing a fixed set of possible contacts \citep{Hauser06usingmotion, stilman2010}.

Most of the papers that followed the work of \citeauthor{conf/iser/BretlRLKA04} have explored alternative formulations to handle the combinatorial issue. Two main directions have been explored. \textbf{On one hand, local optimization of both the root trajectory \Pa and the contact positions $\mathcal{P}_2$} has been used, to trade the combinatorial of the complete problem for a differential complexity, at the cost of local convergence. A complete example of the potential offered by such approaches was proposed \citep{Mordatch:2012:DCB:2185520.2185539} and successfully applied to a real robot \citep{mordatch2015}. To keep reasonable computation times, the method uses a simplified dynamic model for the avatar. Still, the computation time is far from interactive  (about 1 minute of computation for a sequence of 20 contacts).  A similar approach has been considered for manipulation by \cite{gabicciniisrr15}. \citeauthor{DBLP:conf/humanoids/DeitsT14} propose to solve contact planning globally as a mixed integer problem, but only cyclic, bipedal locomotion is considered. 
\citeauthor{dai2014whole} extend the work of \citeauthor{Posa:2014:DMT:2568343.2568352} to discover the contact sequence for landing motions, but need to specify
the contacts manually for more complex interactions.
A major drawback of these optimization based approaches is thus that they only offer local convergence when applied to acyclic contact planning.

\textbf{On the other hand, the two problems \Pa and \Pb might be decoupled} to reduce the complexity. The feasibility and interest of the decoupling is shown by \citeauthor{DBLP:conf/iser/EscandeKMG08} who manually set up a rough guide trajectory (i.e. an ad-hoc solution to $\mathcal{P}_1$). \Pb  is then addressed as the combinatorial computation of a feasible contact sequence in the neighborhood of the guide. A solution can then be found, efficiently when considering quadruped locomotion \citep{kalakrishnan2011learning}, but at the cost of prohibitive computation times (several hours) for constrained scenarios for humanoid. In the latter case, this approach is suboptimal because the quality of the motion is conditioned by the quality of the guide trajectory,  which is not evaluated \textit{a priori}. \citeauthor{Bouyarmane2009} precisely focus on automatically computing a guide trajectory with guarantees of contact feasibility, by extending key frames of the trajectory into whole-body contact configurations in static equilibrium. Randomly sampled configurations are projected into the contact sub-manifold using a generalized inverse kinematics solver, a computationally expensive process (about 15 minutes are required to compute a guide trajectory in the examples presented). Moreover this explicit projection is yet an insufficient condition and does not provide strong guarantees on the feasibility of the path between two key positions in the trajectory. \citeauthor{7140082} also propose a decoupled approach, with a planning phase based on the reachable workspace of the robot limbs, that characterize the ability to create a contact with 
a discretized environment. This planning phase does not account for collisions, involving that replanning is required in case of failure. This approach is efficient 
in the scenarios demonstrated. In extremely constrained cases such 
as the car egress scenario we address here, we believe that including collision constraints in the planning is a requirement.

For completeness, we lastly mention a new kind of approach, recently proposed from the computer graphics field \citep{hamalainen_cpbp_2015}. The authors
use black box physics simulators to perform Model Predictive Control for the motion of a humanoid character, and manage to obtain dynamically consistent motions
at interactive frame rates. While this new approach provides an exciting direction of research, currently the resulting motions
look unnatural, and do not seem applicable to real robots.

It thus appears that, regarding robotics applications, an integrated optimization-based approach \citep{Mordatch:2012:DCB:2185520.2185539}, able to generate a locally-optimal, complete trajectory within minutes, currently outperforms existing planners, decoupled or not. However, recent contributions are able to interpolate dynamically-feasible
trajectories between contact configurations \citep{Hauser2014, herzog2015trajectory, Park116, Carpentier2016}. \citeauthor{Carpentier2016} are able to achieve this with real-time performances on the real robot.
To generate the input contact configurations, there is a need for an efficient contact planner, able to break the combinatorial to generate discrete contact sequences rapidly. 
Such a planner holds the promise of a complete real-time multi-contact locomotion system.

\subsection{Rationale}
This paper presents a pragmatic approach to break the complexity of multi-contact planning. With that objective in mind,
we believe that the separation between the generation of the guide trajectory and the contact sequence is the most promising direction \citep{DBLP:conf/iser/EscandeKMG08}.
However, this direction raises two theoretical questions that remain to be solved, or even to be properly formulated. \\

\noindent \textbf{Regarding $\mathcal{P}_1$}, the guide trajectory must guarantee the existence of a contact sequence to actuate it (This property is related to the controllability of the root actuated by the contact forces). We call this property \textit{true feasibility}. This property has not been studied yet; the only way to validate a waypoint in the trajectory is to explicitly compute the contact locations and forces, which is computationally not reasonable \citep{Bouyarmane2009}, unless the scenario is limited to cyclic, quasi-flat cases \citep{zucker2010optimization}. \\

\noindent \textbf{Regarding $\mathcal{P}_2$}, there are infinite combinations of possible contact sequences for a given root trajectory. The selection of one particular contact sequence with interesting properties (minimum number of contact changes, robustness, efficiency or naturalness) has been studied for cyclic cases \citep{Hauser06usingmotion}, but has not been efficiently applied to cluttered environments (\citeauthor{bouyarmane:lirmm-00777727, DBLP:conf/iser/EscandeKMG08} mostly randomly picked one contact sequence, leading to possibly very tedious transitions).  \\


%~ In this paper, we address both questions.
%~ For each problem, we introduce an original formulation of the related theoretical issues.
%~ Then, we propose a concrete implementation of a solution, where the constraints of the problem are relaxed in favor of efficiency.

The key idea of the paper is the theoritical and practical characterization of the space of \textit{truly feasible} configurations to address both issues.
We first exhibit a low dimensional over-approximation of this space. This approximation is defined geometrically. 
We make the strong assumption that these \textit{geometrically feasible} paths can be efficiently transformed into \textit{truly feasible} paths
in cluttered environments. 
Under this assumption we can take advantage of the decoupling approach.

We first address \Pa by planning \textit{geometrically feasible} paths. We achieve this efficiently thanks to the \textit{reachability condition} that we introduce, that 
does not require explicit contact computation.

Then we address \Pb by extending the path into a \textit{truly feasible} sequence of contact configurations.
This second step requires computing contact configurations verifying a condition for \textit{static equilibrium feasibility}, that requires
explicit contact computation.

This sequential approach is the key to the efficiency of our method, though it
can result in failures because our assumption is not always true. However, we demonstrate empirically the validity of the approach: the high success rate of our method, combined with the low computation times we obtain, allows us to plan (and re-plan upon failure) multi contact sequences at interactive rates.

\subsection{Paper contribution and organization}


By tackling these issues, our method proposes contributions to both problems \Pa and \Pb.
\begin{itemize}
\item First, we propose a theoretical characterization of an efficient approach to sampling-based planning of acyclic contacts;
\item Then we propose a very efficient and general implementation of an acyclic contact planner, the first one compatible with interactive applications;
\item Finally we propose three heuristics for contact generation. They bias the planner towards configurations that are robust regarding static equilibrium, or 
efficient regarding the task being performed.
\end{itemize}

Compared to previous work \citep{Mordatch:2012:DCB:2185520.2185539} our planner does not produce a complete motion, but a discrete sequence of contacts.
Since the interpolation of a complete trajectory between such configurations can be solved within real-time constraints \citep{Carpentier2016}, we claim that our planner
is the fastest multi-contact planning solution to our knowledge.

\begin{figure*}
  \centering
  \begin{overpic}[width=0.8\linewidth]{figures/workflow}
    %~ \put (6.4,1.8) {\normalsize{Request}} 
    \put (30,10.8) {\large{\color{white}\Pa} }
    %~ \put (31.6,2) {\scriptsize{\color{white}RB-RRT}} 
    \put (66.4,10.8) {\large{\color{white}\Pb} }
    %~ \put (72.5,2) {\tiny{\color{white}Contact}} 
  \end{overpic}
  \vspace{-1em}
  \caption{
    Overview of our 2-stage framework. \Pa: Given a path request between the yellow and blue positions, a guide path is computed in the space of truly feasible guides $C_{reach}$. This is achieved by defining a geometric condition, the reachability condition (abstracted here with the transparent cylinders). \Pb: The trajectory is extended into a discrete sequence of contact configurations using an iterative algorithm.}
  \label{fig:framework}
\end{figure*}

In Section~\ref{overview}, we present the general organization of our method. Section~\ref{rbprm} and Section~\ref{sec:contact} present respectively our answer to \Pa and \Pb. In Section~\ref{sec:heuristics}, we present two heuristics used for the selection of a contact configuration. Finally, we propose a complete experimental validation of the planner with three very different kinematic chains (the HRP-2 and HyQ robots, and a three-finger manipulator) in various scenarios,
that complete those presented in the conference paper.
\subsection*{Comparison with our previous work.}
The present paper is an extension of a conference paper to appear in the proceedings of the ISRR'15 conference~\citep{tonneauisrr15}.
The conference paper focuses on the theoretical formulation of the problem, and presents results obtained with virtual avatars.
This extension completes this work with a discussion on the implementation of this theory to real-world robots and problems.
In particular, we introduce a robust balance criterion, designed to ensure the equilibrium of the robot despite bounded error in the contact forces. We also provide details on the HPP implementation
of our contact generation algorithm. Furthermore, our experiments are now applied to real-robot models, namely HRP-2 and HyQ~\citep{semini11hyqdesignjsce}. Appendix~\ref{app:rom} describes the generation process
for the robot.
