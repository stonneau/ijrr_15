% !TEX root =  ../main.tex
\section{From a guide path to a discrete sequence of contact configurations ($\mathcal{P}_2$)}
\label{sec:contact}
In the second phase, we compute a discrete sequence of static equilibrium configurations $\mathbf{Q}^{\overline{0}}$ given a root path
$\mathbf{q}^0(t) : [0,1] \longrightarrow$ \gls{$C_{Reach}^0$}. This contact planner uses a contact generator, used to generate static equilibrium configurations. We first describe the contact planning algorithm, before describing
the contact generator.
%~ Our planner computes guide paths in $C_{Reach}^0$ , an approximation of \gls{$C_{Equil}^0$}.
%~ As an input of this stage, we however assume an \gls{equilibrium feasible} root guide path $\mathbf{q}^0(t) : [0,1] \longrightarrow$ \gls{$C_{Equil}^0$}.
%~ If this is not the case, our planner will fail rapidly, thus allowing replanning, as discussed in Section~\ref{sec:perf}.
%~ We now consider the second problem of computing a discrete sequence of equilibrium configurations $\mathbf{Q}^{\overline{0}}$ along $\mathbf{q}^0(t)$.


%~ In this Section we first describe a single contact-generation process, that is how to generate a contact configuration for a limb, given a
%~ root location.
%~ Then, we propose an iterative algorithm to generate a discrete sequence of contact configurations in static equilibrium.

%~ Our criterion to assert efficiently the static equilibrium of the system
%~ is described in Appendix~\ref{sec:heuristics}.

\subsection{Definition of a contact sequence}
In previous contributions~\citep{DBLP:conf/iser/EscandeKMG08}, a contact plan is defined as a sequence of quasi-static equilibrium configurations
for each contact phase. For instance, a walk cycle would be described by three key configurations: a double-support configuration, a single-support configuration (a contact is broken), and another double-support configuration (a contact is created). 
Our definition of contact plan differs: between two consecutive configurations we allow both a contact break and a contact creation---if they are on the same effector. 
In the previous example, our contact plan would simply consist of the two double-support configurations. 
%However, if the contact broken and the contact created sequentially are not on the same effector, our planner outputs two distinct states.
This representation is sufficient to describe all the contact phases. Furthermore it removes the need to have single-support quasi-static configurations as in the example. 
As shown in the companion video, this allows our framework to produce dynamic motions. 

\subsection{Contact planning algorithm}
Starting from an initial whole-body configuration, we compute a sequence
of whole-body configurations  $\mathbf{Q}^{\overline{0}}$ along the root path $\mathbf{q}^0(t)$.
We first give an intuition of the algorithm, before providing its complete pseudo-code.
%~ The algorithm can be found in Appendix~\ref{app:contact}. Here we provide an intuition of it.
\subsubsection{Algorithm overview}
First, the root path $\mathbf{q}^0(t)$ is discretized into a sequence of $j$ key configurations:  
\begin{equation*}
	\mathbf{Q}^0 = [\mathbf{q}^0_{0}; \mathbf{q}^0_{i}; ..., \mathbf{q}^0_{j-1}]
\end{equation*} 
where $\mathbf{q}^0_{0}$ and $\mathbf{q}^0_{j-1}$ are the start and goal configurations. %To ensure continuity in the contact transition phases,
Each root configuration of $\mathbf{Q}^0$ is then extended into a whole-body configuration such that:
\begin{itemize} 
\item At most one contact is not maintained (\textit{broken}) between two consecutive configurations.
\item At most one contact is added between two consecutive configurations.
\item Each configuration is in static equilibrium.
\item Each configuration is collision-free.
\end{itemize} 


%~ We want to extend the configurations of $\mathbf{Q}^0$ in such a way that continuity is preserved regarding the contact transitions.
%~ To do so, we define an algorithm that, given the current root configuration, and the previous full body configuration, computes a full body configuration 
%~ in $C_{Equil}$ such that contacts are maintained if possible.
%~ The first full body configuration of the sequence is given by the initial state of the robot.

 %~ we propose a recursive mapping $\pi$, for any $0<i<j$:
%~ \begin{equation*}
    %~ \pi\colon\left\{
    %~ \begin{aligned}		
        %~ \mathbf{Q}^0 \in C_{Equil}^0 & \longrightarrow C_{Equil} \\
        %~ %\mathbf{q}^{0}_0 &  \longrightarrow  \mathbf{q}_{start} \\
        %~ \mathbf{q}^{0}_i &  \longrightarrow  g(\mathbf{q}_{i - 1},\mathbf{q}^{0}_i) 
    %~ \end{aligned}
    %~ \right.
%~ \end{equation*} 
%~ $g$ is the method that extends a root configuration into a full-body configuration. At each step, it tries to generate a contact configuration that preserves as much as possible
%~ the previous contacts while allowing for static equilibrium. The objective is thus to characterize $g$.
%~ We initialize the recurrence with $\pi(\mathbf{q}^0_{0}) = \mathbf{q}_0$ the initial configuration of the robot.

%~ The function $g$ is defined independently by $g^k$ for each limb $R^k$. In defining $g^k$, two aspects must be considered. Is the limb $R^k$ in contact? And which criteria is it optimizing? 

\paragraph{Maintaining a contact in the sequence}

%~ Figure~\ref{fig:break_contact} illustrates the contact-persistence strategy.
If kinematically possible, a limb in contact at step $i-1$ remains in contact at step $i$ (Figure~\ref{fig:break_contact}). 
%~ The contact is broken if an inverse-kinematics solver fails to find a collision-free limb configuration that satisfies joint limits. 
%~ The solver is directly provided by the HPP software.
Otherwise the contact is broken and a collision-free configuration is assigned to the limb.
If two or more contacts can't be maintained between two consecutive configurations, one or more intermediate configurations are added, to ensure
that at most one contact is broken between two sequential configurations.
%~ For these steps the root configuration is the same as for the previous step, with the difference that
%~ one faulty contact is repositioned, in the hope that it will not be broken at the next step.

\begin{figure}[t]
\centering
  \begin{overpic}[width=0.9\linewidth]{figures/break_contact}
		\put (0,4) {1} 
		\put (25,4) {2} 
		\put (50,4) {3} 
		\put (76,4) {4} 
		%~ \put (68,58) {3.a)} 
		%~ \put (5,27) {3.b)} 
		%~ \put (37,27) {4.a)} 
		%~ \put (68,27) {4.b)} 
	\end{overpic}
\caption{Contacts are maintained if joint limits and collisions constraints are respected (2). They are broken otherwise(3,4). The green line represents the root path.}
		   \label{fig:break_contact}
\end{figure}

%~ \begin{figure}[t]
%~ \centering
  %~ \begin{overpic}[width=0.6\linewidth]{figures/generate_contact}
		%~ \put (5,58) {1)} 
		%~ \put (37,58) {2)} 
		%~ \put (68,58) {3.a)} 
		%~ \put (5,27) {3.b)} 
		%~ \put (37,27) {4.a)} 
		%~ \put (68,27) {4.b)} 
	%~ \end{overpic}
%~ \caption{Contacts are generated when the configuration is not balanced.}
		   %~ \label{fig:generate_contact}
%~ \end{figure}


%\subsection{Generation of a contact configuration}  
\paragraph{Creating contacts}
Contacts are created using a FIFO approach: we try first to create a contact with the limb that has been contact-free the longest. If the contact creation does not succeeds, the limb is pushed on top of the queue, and will only be tried again after the others. \\ \\
%~ \begin{enumerate}
%~ \item \deladp{At most one contact creation happens between two consecutive steps. }
%~ \item \deladp{A contact is validated if and only if the resulting configuration is in static equilibrium;}
%~ \item \deladp{We use a FIFO approach:  we always try first to create a contact with the limb that has been contact-free the longest. If the contact creation
%~ was not successful for a limb, the limb is pushed on top of the queue, and will only be tried again after the others.}
%~ \end{enumerate}

%~ \deladp{However, in practice the planner is successful in the large majority of cases, as discussed in Section~\ref{sec:perf}.}


% !TEX root =  ../main_tro.tex
\subsubsection{Pseudo-code of the Algorithm}
\label{app:contact}


%~ First, we define an abstract structure State,
%~ that describes a contact configuration.
%~ The use of queues allows a FIFO approach regarding the order 
%~ in which contacts are tested: we try to replace older contacts first when necessary.
%~ Thus the algorithm is deterministic even though it can handle acyclic motions. \\
%~ 
%~ \begin{lstlisting}]
%~ Struct Limb
%~ {
    %~ // Limb Configuration
    %~ Configuration qk;
    %~ // Effector position in
    %~ // world coordinates
    %~ vector6 pk;
%~ };
%~ 
%~ Struct State
%~ {
    %~ // root location
    %~ Configuration q0;
    %~ // List of limbs not in contact
    %~ queue<Limb> freeLimbs;
    %~ // List of limbs in contact
    %~ queue<Limb> contactLimbs;
%~ };
%~ \end{lstlisting}

From the start configuration, given as an input by the user,
we create the initial state $s0$.
Algorithm~\ref{alg:interpolate}  is then called with $s0$, as well as the discretized path 
$\mathbf{Q}^0$, as input parameters.

\begin{algorithm}[!tbp]
\caption{Discretization of a path} \label{interpolate}
	\begin{algorithmic}[1]
	%~ \Function{GenerateConfiguration}{}
	\Function{Interpolate}{$s0$,$\mathbf{Q}^0$, $MAX\_TRIES$}
		\State $states \gets (s0)$ \Comment{List of states intialized with $s0$}
		\State $nb\_fail \gets 0$ 
		\State $i \gets 1;$ \Comment{Current index in the list}
		\While {$i < length(\mathbf{Q}^0)$}
			\State $pState \gets last\_element(states)$
			\State $s \gets$ \textsc{GenFullBody}$(pState, element(\mathbf{Q}^0,i))$
			\If {$s \not= 0 $}
				\State $nb\_fail \gets 0$
				\State $i \gets i+1$
				\State \textbf{return} $\mathbf{q}^{0}$
			\Else
				\State $nb\_fail \gets nb\_fail + 1$
				\If {$nb\_fail == MAX\_TRIES$}
					\State \textbf{return} $FAILURE$
				\EndIf				
				\State $s \gets $\textsc{IntermediateContactState}$(pState)$
			\EndIf
			\State $push_back(states, s)$
		\EndWhile
		\State \textbf{return} $states$
	\EndFunction
\end{algorithmic}
\label{alg:interpolate}
\end{algorithm}

At each step, \textsc{GenFullBody} is called with the previous state as a parameter, as well
as a new root configuration. \textsc{GenFullBody} returns a new contact configuration, if it succeeded
in computing a configuration with only one contact switch occurring.
Otherwise, the method \textsc{IntermediateContactState} is called.
It repositions one end effector (either a free limb, or the oldest active contact) towards a new contact position if possible.
This repositioning allows to increase the odds that the contact can be maintained at the next step.
The method \textit{last\_element} returns the last element
of the list, \textit{element} returns the element contained by a list at a given index, and \textit{push\_back} inserts an element at the end of a list.
Algorithm~\ref{alg:pi} gives the pseudo code for \textsc{GenFullBody}.

\begin{algorithm}[!tbp]
\caption{Full body contact generation method} \label{interpolate}
	\begin{algorithmic}[1]
	%~ \Function{GenerateConfiguration}{}
	\Function{\textsc{GenFullBody}}{$pState$,$\mathbf{q}^0$}
		\State State $newState$
		\State $newState.q0 = \mathbf{q}^0$
		\State $newState.freeLimbs = pState.freeLimbs$
		\State /*First try to maintain previous contacts*/
		\State $nbContactsBroken = 0$
		\For {\textbf{each} Limb $k$ in $pState.contactLimbs$}
			\If {$!$\textsc{MaintainContact}$(pState,\mathbf{q}^0,k)$}
				\State $nbContactsBroken += 1$
				\If {$nbContactsBroken > 1$}				
					\State \textbf{return} $NULL$
				\EndIf				
				\State $push(newState.freeLimbs,k)$
			\Else 					
				\State $push(newState.contactLimbs,k)$
			\EndIf
		\EndFor
		\For {\textbf{each} Limb $k$ in $pState.freeLimbs$}
			\If {\textsc{GenerateContact}$(\mathbf{q}^0,k)$}	
				\State $push(newState.contactLimbs,k)$
				\State $remove(newState.freeLimbs,k)$		
				\State \textbf{return} $newState$
			\EndIf
		\EndFor
		\If {\textsc{IsInStaticEquilibrium}$(newState)$}
			\State \textbf{return} $newState$
		\Else
			\State \textbf{return} $NULL$
		\EndIf
	\EndFunction
\end{algorithmic}
\label{alg:pi}
\end{algorithm}

The method \textsc{MaintainContact}$(pState,\mathbf{q}^0,k)$ performs inverse kinematics to reach the previous contact position for the Limb.
If it succeeds, the new limb configuration is assigned to $k$. If it fails, a random collision free configuration is assigned to $k$.

The method \textsc{IsInStaticEquilibrium} returns whether a given state is in static equilibrium.

The pseudo code for the method \textsc{IntermediateContactState} is given by Algorithm~\ref{alg:repo}.


\textsc{GenerateContact}$(\mathbf{q}^0,k)$ is a call to the contact generator presented in the following Section~\ref{sec:single_contact}.
 It generates a contact configuration in static equilibrium, and assigns the corresponding configuration to $k$.
If it fails, $k$ remains unchanged if it is collision free, otherwise it is assigned a random collision free configuration.



\begin{algorithm}[!tbp]
\caption{Adds or repositions a contact for one limb} \label{interpolate}
	\begin{algorithmic}[1]
	\Function{IntermediateContactState}{$state$}
		\State $i=0$
		\While {$i<length(states.freeLimbs)$}
			\State Limb $k = pop(states.freeLimbs)$
			\If {\textsc{GenerateContact}$(state.q0,k)$}	
				\State $push(newState.contactLimbs,k)$			
				\State \textbf{return}
			\Else
				\State $i+=1$
				\State $push(states.freeLimbs,k)$		
			\EndIf
		\EndWhile
		\State $i=0$
		\While {$i<length(states.contactLimbs)$}
			\State Limb $k = pop(states.contactLimbs)$
			\State Limb $copy = k$
			\State $i+=1$
			\If {\textsc{GenerateContact}$(state.q0,k)$}	
				\State $push(newState.contactLimbs,k)$			
				\State \textbf{return}
			\Else
				\State $push(newState.contactLimbs,copy)$	
			\EndIf
		\EndWhile
		/*Fails if impossible to relocate any effector*/
		\State \textbf{return} $FAILURE$
	\EndFunction
\end{algorithmic}
\label{alg:repo}
\end{algorithm}

%~ \section{Derivation of the manipulability ellipsoid}
%~ \label{app:manipulability}
%~ 
%~ Again, we assume that $\mathbf{J}$ is full rank. We discard the $k$ indices, and write the pseudo-inverse of $\mathbf{J}$ as $\mathbf{J}^{\dagger}$.
%~ \begin{eqnarray*}
%~ \mathbf{\dot{p}} & =  & \mathbf{J} \dot{\mathbf{q}} \\ 
%~ \mathbf{\dot{q}} & =  & \mathbf{J}^{\dagger} \dot{\mathbf{p}} \\ 
%~ \mathbf{\dot{q}}^T & =  & \dot{\mathbf{p}}^T \mathbf{J}^{\dagger T} \\ 
%~ \mathbf{\dot{q}}^T\mathbf{\dot{q}} & =  & \dot{\mathbf{p}}^T \mathbf{J}^{\dagger T} \mathbf{J}^{\dagger} \dot{\mathbf{p}}\\ 
%~ \end{eqnarray*}
%~ 
%~ Then, the equality $\mathbf{J}^{\dagger T} \mathbf{J}^{\dagger} = (\mathbf{J}\mathbf{J}^T)^{-1}$ follows from the SVD decomposition of each term
%~ ~\citep{ben2003generalized}.


\subsection{Contact generator}
\label{sec:single_contact}

\begin{figure*}
  \centering
  \begin{overpic}[width=0.8\linewidth]{figures/contact_gen}
		\put (1,1) {a} 
		\put (22,1) {b} 
		\put (42,1) {c} 
		\put (62,1) {d} 
		\put (83,1) {e} 
	\end{overpic}
  \caption{Generation of a contact configuration for the right leg of HRP-2. (a): Selection of reachable obstacles. (b): Entries of the limb samples database (with $N = 4$). (c): With a proximity query between the octree database and the obstacles, configurations too far from obstacles are discarded. (d): The best candidate according to a user-defined heuristic $h$ is chosen. (e): The final contact is achieved using inverse kinematics.}
  \label{fig:contact_gen}
\end{figure*}

Given a configuration of the root and the list of effectors that should be in contact, the contact generator computes the configuration of the limbs such that contacts are properly satisfied and the robot is in static equilibrium:

\begin{equation}
\label{eq:contact_gen}
	\mathbf{q}^{\overline{k}}  \longrightarrow \mathbf{q}^k, (\mathbf{q}^{k} \oplus \mathbf{q}^{\overline{k}}) \in  C_{Equil} \textrm{ \textbf{and}}\ \mathbf{q}^k \in  C_{Contact}^k 
\end{equation}

In previous works~\cite{DBLP:conf/iser/EscandeKMG08,Bouyarmane2009}, the generation of contact is typically implemented by randomly sampling configurations and projecting the whole robot configuration onto the closest surfaces with an inverse kinematics solver.
In case of failure of the projection, the process would randomly iterate.


We propose two modifications of this general algorithm principle.
First our contact generator handles each limb $R^k$ independently.
By handling each limb separately, we reduce the complexity of the generation of contact configurations.
This is made possible thanks to the reachability condition in $\mathcal{P}_1$ that produces a root path that we can afford not to modify in $\mathcal{P}_2$, and because we allow both a contact break and a contact creation between two consecutive configurations of the contact sequence.
Second, we rely on off-line generation of configuration candidates.
%~ Contact generation is typically addressed by randomly sampling a limb configuration, before projecting the effector onto the closest surface with an inverse kinematics solver.
%~ This process is repeated until either a solution for Eq.~(\ref{eq:contact_gen}) is found, or the generator fails, according to a user-defined maximum number of trials.
%~ We favor a modified implementation of this naive approach, more computationally efficient, introduced in our previous work~\citep{Tonneau2014}.



%~ Given a configuration $\mathbf{q}^{\overline{k}}$ of the root and all the limbs but $R^k$, we look for a limb configuration $\mathbf{q}^k$ such that
%~ $R^k$ is in contact, and not colliding (neither with parts of the robot nor with the environment).
%~ While exhibiting analytically a $\mathbf{q}^{k}$ does not seem tractable, we can iteratively try to generate one as follows:
%~ \begin{enumerate}
%~ \item Generate randomly a collision-free limb configuration;
%~ \item Project the end-effector onto the closest surface with inverse kinematics;
%~ \item If a valid solution is found, stop. Otherwise repeat from step 1.
%~ \end{enumerate}


%~ We favor a modified implementation of this naive approach, more computationally efficient, introduced in our previous work~\citep{Tonneau2014}.

We define $C_{Contact}^{\epsilon} \supset C_{Contact}$ as the set of configurations such that the minimum distance 
between an effector and an obstacle is less than $\epsilon \in \mathbb{R}$.
We then apply the following steps:
\begin{enumerate}
\item Generate off-line $N$ valid sample limb configurations $\mathbf{q}^k_i,  0 \leq i < N$ (We choose $N=10^4$);
\item Using the end-effector positions $\mathbf{p}(\mathbf{q}^k_i)$ as indices, store each sample in an octree data structure;
\item At runtime, when contact creation is required, intersect the octree and the environment to retrieve the list of samples $S \subset C_{Contact}^{\epsilon}$ close to contact (Figure~\ref{fig:contact_gen} (b) and (c));
\item Use a user-defined heuristic $h$ to sort $S$;
\item If $S$ is empty, stop (failure). Else select the first configuration of $S$. Project it onto contact using inverse kinematics. (Figure~\ref{fig:contact_gen} (d) and (e));
\item If Eq.~\ref{eq:contact_gen} is verified, stop (success). Otherwise remove the element from $S$ and go to step 5.
\end{enumerate}

%~ The reader is referred to our previous work for an extensive discussion on the benefits of this approach, and the optimal choice 
%~ of the parameter $N$~\citep{Tonneau2014}. Our criterion to assert efficiently the static equilibrium of the robot, as well as the heuristics $h$ are detailed in Appendix~\ref{sec:heuristics}.
In all our experiments, the heuristic $h$ is implemented as a variation of a manipulability-based heuristic~\cite{Yoshikawa1984}. The manipulability is a real number that quantifies how 
``good'' a configuration is to perform a given task, based on the analysis of the Jacobian matrix. With such heuristics, a configuration can be chosen because it is far from singularities, and thus allows mobility in all directions. On the contrary, it can be chosen because it is particularly efficient to exert a force in a desired direction. In our experiments, the former solution is usually chosen for computing leg contacts, while the latter is used for computing hand contacts. We recall the manipulability measure and its derivatives in Appendix~\ref{sec:heuristics}.

Finally, to verify that a configuration is in static equilibrium, we use a new robust LP formulation. It replaces the computationally inefficient double description
approach used in our previous work~\cite{tonneauisrr15}, and presented in the following Section~\ref{sec:Equil}.



%~ TODO: $h$ in appendix, the static equilibrium is here ~\ref{sec:Equil}

% !TEX root =  ../main.tex
\section{Heuristics for contact selection}
\label{sec:heuristics}
\subsection{A heuristic for robust static equilibrium}
The planner is designed so that any generated contact configuration is in static equilibrium.
%~ Equilibrium is of course critical in legged locomotion. For this reason 
We are interested in a robust 
criterion, that ensures that the robot remains in equilibrium in a real-world application, regardless of perception and control uncertainties.

We first give a linear program (LP) that verifies whether a contact configuration allows for static equilibrium.
From this formulation we derive a new LP that quantifies the robustness of the equilibrium to uncertainties in the contact forces.
In turn, from this value we can either choose the most robust candidate, or set a threshold on the required robustness. While the presented LP is original, it is based on an analysis of the problem that we proposed in \citep{Prete2016}, where the interested reader can find more details.


\subsubsection{Conditions for static equilibrium:}
We first define the variables of the problem, for $e$ contact points, expressed in world coordinates:
\begin{itemize}
\item $\mathbf{c} \in \mathbb{R}^3$ is the robot center of mass (COM);
%~ \item $\mathbf{L}  \in \mathbb{R}^3 $ is the angular momentum at the COM;
\item $m \in \mathbb{R}$ is the robot mass;
\item $\mathbf{g} = [0,0,-9.81]^T$ is the gravity acceleration;
\item $\mu$ is the friction coefficient;
\item for the i-th contact point $1 \leq i \leq e$:
	\begin{itemize}
	\item $\mathbf{p_i}$ is the contact position;
	\item $\mathbf{f_i}$ is the force applied at $\mathbf{p_i}$;
	\item $\mathbf{n}_i,\mathbf{t}_{i1},\mathbf{t}_{i2}$ form a local Cartesian coordinate system centered at $\mathbf{p_i}$. $\mathbf{n}_i$ is aligned
	with the contact surface normal, and the $\mathbf{t}_i$s are tangent vectors.
	\end{itemize}
\end{itemize}

According to Coulomb's law, the nonslipping condition is verified if all the contact forces lie in the friction cone defined by the surface.
As classically done, we linearize the friction cone in a conservative fashion with a pyramid, described by four generating rays of unit length. We choose for instance:
\begin{equation*}
\mathbf{V}_{i} = \mat{\mathbf{n}_{i} + \mu \mathbf{t}_{i1} & \mathbf{n}_{i} -\mu \mathbf{t}_{i1} & \mathbf{n}_{i} + \mu \mathbf{t}_{i2} & \mathbf{n}_{i} - \mu \mathbf{t}_{i2}}^T
\end{equation*}

Any force belonging to the linearized cone
can thus be expressed as a positive combination of its four generating rays.
%~ \deleted{Therefore we can express the nonslipping constraint on $\mathbf{f}_i$ as:}

\begin{equation*}
\forall i  \qquad  \exists \bm{\beta}_i \in \mathbb{R}^{4} : \bm{\beta}_i \ge 0 \text{ and } \mathbf{f}_{i} = \mathbf{V}_{i} \bm{\beta}_i,
\end{equation*}
where $\bm{\beta}_i$ contains the coefficients of the cone generators.
We can then stack all the constraints to obtain:
\begin{equation}\label{eq:gen}
\exists \bm{\beta} \in \mathbb{R}^{4e} ,  \bm{\beta} \ge 0 \text{ and } \mathbf{f} = \mathbf{V} \bm{\beta},
\end{equation}
where $\mathbf{V} = \diag{ \{\mathbf{V}_1, \dots, \mathbf{V}_e\} }$, and $\mathbf{f} = (\mathbf{f}_0,...,\mathbf{f}_e)$.

From the Newton-Euler equations, to be in static equilibrium the contact forces have to compensate the gravitational forces:


%~ \begin{align}
%~ \underbrace{\mat{m (\ddot{\mathbf{c}} - \mathbf{g})  \\ m \mathbf{c} \times (\ddot{\mathbf{c}} - \mathbf{g}) + \dot{\mathbf{L}}}}_\mathbf{w}
%~ = 
%~ \underbrace{
%~ \mat{\mathbf{I}_3 & \dots & \mathbf{I}_3 \\
%~ \hat{\mathbf{p}}_1 & \dots & \hat{\mathbf{p}}_e} \mathbf{V}
%~ }_\mathbf{G}
%~ \bm{\beta},
%~ \end{align}
%~ where $\mathbf{w}\in \Rv{6}$ is the so-called gravito-inertial wrench (GIW) \citep{qiu:dhm:2011, Caron2015} and $\hat{\mathbf{p}} \in \R{3}{3}$ is the cross-product matrix associated to $\mathbf{p}$.

%~ Static balance assumes zero acceleration thus we can write:
\begin{align} \label{eq:new_eul}
\underbrace{
\mat{\mathbf{I}_3 & \dots & \mathbf{I}_3 \\
\hat{\mathbf{p}}_1 & \dots & \hat{\mathbf{p}}_e} \mathbf{V}
}_\mathbf{G} \bm{\beta}, = 
\underbrace{\mat{\mathbf{0}_{3\times 3} \\ m \hat{\mathbf{g}}}}_{\mathbf{D}} \mathbf{c} + 
\underbrace{\mat{-m\mathbf{g} \\ \mathbf{0}}}_{\mathbf{d}}
\end{align}
where $\hat{\mathbf{x}} \in \R{3}{3}$ is the cross-product matrix associated to $\mathbf{x}$.
%~ The right hand side of \eqref{eq:new_eul} is independent of $\mathbf{c}^z$ because of the cross product matrix $\hat{\mathbf{g}}$ inside $\mathbf{D}$, so we can rewrite it as: \adnote{It is not really necessary to show that equilibrium is independent of the com altitude here, anyway the com is given, not a variable, so it does not change the size of the problem}
%~ \begin{align} \label{eq:new_eul2d}
%~ \mathbf{w}_0 = \mathbf{D}^{xy} \mathbf{c}^{xy} + \mathbf{d}
%~ \end{align}

If there exists a $\bm{\beta}^*$ satisfying \eqref{eq:gen} and \eqref{eq:new_eul}, it means that the configuration is in static equilibrium.
The problem can then be formulated as an LP:

\begin{equation} \label{eq:lin_prog} \begin{aligned}
\find \quad & \bm{\beta} \in \Rv{4e} \\
%~ \st \quad &\mathbf{G} \bm{\beta} = \mathbf{D}^{xy} \mathbf{c}^{xy} + \mathbf{d} \\
\st \quad &\mathbf{G} \bm{\beta} = \mathbf{D} \mathbf{c} + \mathbf{d} \\
& \bm{\beta} \ge 0 \\
\end{aligned} \end{equation}

\subsubsection{Formulation of a robust LP:}
Let $b_0 \in \mathbb{R}$ be a scalar value. We now define the following LP:

\begin{equation} \label{eq:lin_prog_rob} \begin{aligned}
\find \quad & \bm{\beta} \in \Rv{4e}, b_0 \in \Rv{} \\
\minimize  \quad & -b_0 \\
\st \quad &\mathbf{G} \bm{\beta} = \mathbf{D} \mathbf{c} + \mathbf{d} \\
%~ \st \quad &\mathbf{G} \bm{\beta} = \mathbf{D}^{xy} \mathbf{c}^{xy} + \mathbf{d} \\
& \bm{\beta} \ge b_0 \bm{1}\\
\end{aligned} \end{equation}

We observe that if $b_0$ is positive then \eqref{eq:lin_prog} admits a solution, and $b_0$ is proportional to the minimum distance of the contact forces to the boundaries of the friction cones.
If $b_0$ is negative, the configuration is not in static equilibrium, and $b_0$ indicates ``how far'' from equilibrium the configuration is. We thus use $b_0$ as a measure of robustness.

In our implementation, rather than solving directly \eqref{eq:lin_prog_rob}, we solve an equivalent problem of smaller dimension that we get by taking the dual of \eqref{eq:lin_prog_rob} and eliminating the Lagrange multipliers associated to the inequality constraints:
\begin{equation} \label{eq:dual} \begin{aligned}
\find \quad & \bm{\nu} \in \Rv{6}\\
\maximize  \quad & -(\mathbf{D} \mathbf{c} + \mathbf{d})^T \nu \\
\st \quad &\mathbf{G}^T \bm{\nu} \ge 0 \\
& \mathbf{1}^T \mathbf{G}^T \bm{\nu} = 1 \\
\end{aligned} \end{equation}

Indeed, from Slater's conditions \citep{Boyd:2004:CO:993483}, we know that the optimal values of an LP and its dual are equal. Thus the optimal value of the LP~\eqref{eq:dual} is indeed the optimal $b_0$.

%~ In conclusion, the LP~\eqref{eq:dual} provides a method to check rapidly static equilibrium, and returns a robustness value.
%~ This robustness criterion can be immediately transformed into a heuristic, and eventually be weighted with others.
%~ Another option to enforce robustness is to set a minimum acceptable value for $b_0$, but as a result this conservative approach discards valid solutions.

