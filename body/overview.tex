% !TEX root =  ../main.tex
\section{Overview}
\label{overview}

%~ \begin{table*}
%~ \centering
%~ \begin{tabular}{ c | >{\arraybackslash}m{200pt} | c}
   %~ Robot configuration set &  \centering Definition: $\mathbf{q} \in$ set $\Rightarrow$?  & Root configuration set\\
 %~ \hline
 %~ \hline
   %~ $C_{contact}$ & $\mathbf{q}$ collision free; at least one end effector in contact. &  $\mathbf{q}^{0} \in C_{contact}^0$ is ``contact feasible''.  \\
 %~ \hline
   %~ $C_{Equil} \subset C_{contact} $ & $\mathbf{q}$  in static equilibrium &  $\mathbf{q}^{0} \in C_{Equil}^0$ is ``equilibrium feasible''. \\
 %~ \end{tabular}
%~ \caption{TODO.}
%~ \label{tab:glossary}
%~ \quad
%~ \end{table*}

Figure~\ref{fig:framework} illustrates our workflow.
\Pa and \Pb are addressed in a sequential fashion: we first plan a guide path with the \textit{reachability condition}, before
extending it into a sequence of contact configurations in static equilibrium.
In this Section we describe this workflow and give some notations used throughout the paper. Some key concepts are introduced,
the definitions of which can be found in the glossary at the end of this paper.

%
\begin{figure*}
  \centering
  \begin{overpic}[width=0.8\linewidth]{figures/contact_gen}
		\put (1,1) {a} 
		\put (22,1) {b} 
		\put (42,1) {c} 
		\put (62,1) {d} 
		\put (83,1) {e} 
	\end{overpic}
  \caption{Generation of a contact configuration for the right leg of HRP-2. (a): Selection of reachable obstacles. (b): Entries of the limb samples database (with $N = 4$). (c): With a proximity query on the octree database, configurations too far from obstacles are eliminated. (d): The best candidate according to a user-defined heuristic $h$ is chosen. (e): The final contact is achieved using inverse kinematics.}
  \label{fig:contact_gen}
\end{figure*}

\subsection{Computation of a guide path --- \Pa}
We first consider the problem of planning a relevant guide path. The objective is to compute a path of root configurations that will allow contact creation. A desired requirement is to preserve the completeness of the planner: it should be able to explore any valid guide path, but at the same time, any computed guide path must lead to a sequence of contact configurations in static equilibrium.
The sets we introduce are illustrated in Figure~\ref{fig:dedefeas}.

\begin{figure}[t]
\centering
  \begin{overpic}[width=1\linewidth]{figures/2D_feas}
		\put (15,) {$C_{Equil}^0$      $\subset$} 
		\put (47,) {$C_{Contact}^0$ $\approx$ } 
		\put (76,) {$C_{Reach}^0$} 
		%~ \put (5,27) {3.b)} 
		%~ \put (37,27) {4.a)} 
		%~ \put (68,27) {4.b)} 
	\end{overpic}
\caption{Illustration of several root configurations sets used in this paper in a 2D scene. Obstacles are violet, and units are in meters. To show the sets in a 2D representation, all the rotational joints of HRP-2 are locked in the shown configuration, such that a torso configuration
is only described by two positional parameters (x and y). $C_{Equil}^0$ is included in $C_{Contact}^0$. $C_{Reach}^0$ approximates $C_{Contact}^0$. Depending on a parametrization, we can obtain $C_{Contact}^0 \subset C_{Reach}^0$. Considering the configurations around the top obstacle, we can observe a dramatic
divergence between  $C_{Equil}^0$  and $C_{Contact}^0$ when the problem is not \textit{\gls{cluttered}}.}
		   \label{fig:dedefeas}
\end{figure}

More precisely, a root configuration is \gls{equilibrium feasible} if and only if there exists at least one joint configuration such that the robot satisfies two conditions:
\begin{description}
\item[C1]: is not in collision with the environment, 
%~ \item[C2] has at least one end-effector in contact with the environment, 
\item[C2]: is in static equilibrium. 
\end{description}
A requirement for C2 is the condition C3: at least one end-effector must be in contact with the environment.
C1 and C3 define \glslink{contact feasible}{\textit{contact feasibility}}, and can be efficiently approximated even without computing the contact points. C2 defines \glslink{equilibrium feasible}{\textit{equilibrium feasibility}}, that we describe with the set $C_{Equil}^0$ (Figure~\ref{fig:dedefeas} -- Green). C2 is difficult to approximate and requires the computation of the contact points, as well as the resolution of a linear program to compute the contact forces~\citep{Prete2016}. 

We make the assumption that for the \textit{\gls{cluttered} contact planning problem} (that we define as admitting a solution where every configuration contains at least one contact
is \gls{quasi-flat}) verifying C2 is not critical for $\mathcal{P}_1$.
This means that root configurations satisfying C1 and C3 also satisfy C2. For this reason we neglect C2 during the resolution of \Pa, and we consider it only when solving \Pb.

We define $C_{Contact}^0$ (Figure~\ref{fig:dedefeas} -- Orange) as the set of \gls{contact feasible} root configurations (that can satisfy C1 and C2).
Planning in $C_{Contact}^0$ boils down to planning in $\mathbb{R}^r$, which has an acceptable practical complexity.

An intuitive description of \gls{contact feasible} configurations is ``close, but not too close'': close, because a contact surface must be partially included in the reachable workspace of the robot (represented for the right leg in Figure~\ref{fig:contact_gen}--a); not too close, because the robot must avoid collision.
We approximate $C_{Contact}^0$ with a set $C_{reach}^0$, defined by the \textit{reachability condition}, a geometrical criterion based on this intuitive description (Figure~\ref{fig:dedefeas} -- Red).
Our objective is to find a root path such that the root (scaled by a factor $s \geq 1$) is not in collision, while the reachable workspace is in collision with the environment.
In Section~\ref{rbprm} we detail how  $C_{reach}^0$ is designed and sampled to compute a guide path with a dedicated planner, the Reachability-Based RRT---RB-RRT--- (Figure~\ref{fig:framework}---\Pa).

\subsection{Generating a discrete sequence of contact configurations --- \Pb}

The second stage extends the guide path into a sequence of contact configurations (Figure~\ref{fig:framework}--\Pb).
With a fixed root position, the dimensionality of the 
contact generation problem is reduced to the number of degrees of freedom of the considered limb. In other words, we consider each limb as a manipulator attached to the root, and select the most relevant contact from a database of precomputed configurations, which is independent from the environment (Figure~\ref{fig:contact_gen}).

From a given start configuration, the planner proceeds in an iterative fashion along the discretized path: given a new root configuration, an inverse-kinematics solver 
is used to maintain the contacts that were previously existing. Possibly, some of these contacts cannot be maintained (because of joint limits or collisions).
Then the contacts are broken. Conversely, new contacts are created to ensure the static equilibrium of the robot.
The algorithm is designed so that only one contact can be created or broken between two successive configurations. While it does not provide guarantees that the interpolation between two configurations is achievable, it appears empirically as a reasonable heuristic.
Details are presented in Section~\ref{sec:contact}. 

There is a combinatorial issue in the choice of which contact to create,
and where to create it. This complexity is avoided by selecting the ``most relevant'' contact configurations with user-defined heuristics. In Section~\ref{sec:heuristics} we present extensively those heuristics that have been used in our experiments.


%~ To create a contact, we consider each limb of the robot as a manipulator arm attached to the root. We store a database of configurations for each manipulator. Figure~\ref{fig:contact_gen} -- 2 presents a few configurations for the right arm of the robot. The configurations in the database which are close to contact are considered (Figure~\ref{fig:contact_gen} -- 2 and 3). Among the candidates, 3 criteria are considered to choose the most relevant: the ideal candidate is free of collision, allows to maintain static balance if necessary, and  contributes to the motion in an optimal way. We call the measure of this contribution task efficiency: it represents the ability the limb has to exert a force compatible with the direction of motion. In Figure~\ref{fig:contact_gen} -- 4, the black arrow represents the direction of motion. Among the candidates, the most relevant to achieve a vertical motion is chosen, and projected on the contact surface using an inverse kinematics solver (Figure~\ref{fig:contact_gen} -- 5). The task efficiency is measured using the Extended FORce Transmission ratio (EFORT), proposed by ~\cite{Tonneau:2014:TEC:2619648.2619652}.

\subsection{Notation conventions and definitions} \label{notations}

A vector  $\mathbf{x}$ is denoted with a bold lower-case letter.
A matrix $\mathbf{A}$ is denoted with a bold upper-case letter.
A set $C$ is denoted with an upper-case italic letter.
Scalar variables and functions are denoted with lower-case italic letters, such as
$r$ or $f(\textbf{x})$.

%\subsubsection{Robot description}
%\label{def}

%~ \begin{figure}
  %~ \centering
  %~ \begin{overpic}[width=0.8\linewidth]{figures/character}
    %~ \put (7,29) {\small{$\in \mathbf{q}^1$}}
    %~ \put (7,25) {\small{effector}}
  %~ \end{overpic}
  %~ \caption{
    %~ Left: Robot in a rest configuration. The right arm is denoted as the limb $R^1$. Each colored dot represents a degree of freedom around an axis. Right: Volumes of the robot. The red geometry denotes $W^0$ and must remain collision-free. The green spheres are the reachable workspace of each limb, the  $W^k$.}
  %~ \label{fig:character}
%~ \end{figure}

\medskip
\textbf{A robot} is a kinematic chain $R$, comprising \mbox{$n + r$} degrees of freedom (DOFs), with $r \geq 6$. $r$ denotes the world coordinates of the root of the robot,
as well as extra DOFs not related to the limbs (For instance, HRP-2 has two extra DOFS in the torso, such that we have $r=8$. 
$R$ is composed of $l$ limbs $R^k, 1 \leq k \leq l$, attached to a root.
It is described by a configuration $\mathbf{q} \in  \mathbb{R}^r \times \mathbb{R}^n$.
We define some relevant projections of $\mathbf{q}$:
\begin{itemize}
	\item $\mathbf{q}^k$ denotes the configuration (a vector of joint values) of the limb $R^k$; % (Figure~\ref{fig:character});
	\item $\mathbf{q}^{\overline{k}}$ denotes the vector of joint values of R \textbf{not} related to $R^k$. We define for convenience \mbox{$\mathbf{q}= \mathbf{q}^k \oplus \mathbf{q}^{\overline{k}}$}; %Abusively, we denote by $\mathbf{q}^0 \oplus \mathbf{q}^k$ any configuration respecting $\mathbf{q}^0$ and $\mathbf{q}^k$;
	\item $\mathbf{q}^{0}\in \mathbb{R}^r$ denotes the position and orientation of the root of the robot $R$.
\end{itemize}
%Finally, $ \mathbf{q}_{start}$ and  $\mathbf{q}_{goal}$ are the start and goal configurations given as an input of our problem.

%\subsubsection{Environment and other volumes} 
%\label{sec:A}
\medskip
The volume encompassing the trunk of the robot is denoted $W^0$
 %~ (Figure~\ref{fig:character}-right: central cylinder). 
 The reachable workspace of a limb $R^k$ is denoted $W^k$ %(Figure~\ref{fig:character}-right: the four ellipses). 
\begin{equation}
  W^k = \left\{ {\mathbf{x} \in \mathbb{R}^3: \exists \, \mathbf{q}^k \in C^k_{jl}, \mathbf{p}^k(\mathbf{q}^k) = \mathbf{x} } \right\},
\end{equation}
%~ \adnote{in this definition you don't account for joint limits}
where $\mathbf{p}^k$ denotes the end-effector position of $R^k$ (translation only) for $\mathbf{q}^0$ being the null displacement, and  $C^k_{jl}$ is the space
of admissible limb joint configurations. We also define $W = \bigcup_{k=1}^{l}W^k$, and
$W^k(\mathbf{q}^{0})$ (for $1 \leq k \leq l$) as the volume $W^k$ translated and rotated by the rigid displacement~$\mathbf{q}^{0}$.

\medskip
\textbf{The environment} $O$ is defined as the union of the obstacles $O_i$ that it contains.
%~ Finally, we define $dist(\mathbf{x}, O)$ as the minimal distance between a point $\mathbf{x}$ and the closest surface of the environment.
