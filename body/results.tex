% !TEX root =  ../main.tex
\section{Results}
\label{sec:results}
In this Section we present some of the results obtained with our planner. The complete sequences computed are shown in the companion video, TODO.
Specifically, we demonstrate the planner for two really different robots, in a large variety of environments: the humanoid HRP-2 and the quadruped HyQ.
For each scenario we indicate the chosen heuristics. We also provide a performance analysis, which shows that the planner is compatible with interactive applications,
and present the success rates obtained in each scenario. Moreover, we demonstrate the interest of our robustness criterion in the different computed poses.
Finally, a last example suggests possible applications to dexterous manipulation.

In our previous work~\citep{tonneauisrr15} additional results are demonstrated with various virtual avatars (Figure~\ref{fig:robots_old}).
In this extension we choose to focus on actual robots. We invite the interested reader to watch the ISRR video\endnote{https://www.youtube.com/watch?v=LmLAHgGQJGA}, and 
to refer to the previous paper for a discussion on these results.

We say that the planning is interactive when the computation time for one step is lesser than the
time to execute it. We arbitrarily approximate this time to one second.

\begin{figure}[t]
\centering
  \begin{overpic}[width=1\linewidth]{figures/robots_old}
		%~ \put (5,58) {1)} 
		%~ \put (37,58) {2)} 
		%~ \put (68,58) {3.a)} 
		%~ \put (5,27) {3.b)} 
		%~ \put (37,27) {4.a)} 
		%~ \put (68,27) {4.b)} 
	\end{overpic}
\caption{Virtual avatars in various scenarios demonstrated in our conference paper.}
		   \label{fig:robots_old}
\end{figure}

\subsection{Description of the scenarios}
In all the scenarios considered, the formulation of the problem is always the same:
a start and goal root placements are provided as an input of the scenario.
The framework computes the initial contact configuration, and outputs a sequence of contact configurations connecting it to the goal.
In each scenario we detail the parameters chosen: the heuristics, and the constraints on the reachable workspaces (for instance in all the scenarios,
the reachable workspaces of the legs of HRP-2 are always required to intersect with the environment). 
A companion video available at TODO \url{http://youtu.be/LmLAHgGQJGA} (anonymous link) displays the complete contact sequence obtained in all these scenarios.

% \subsubsection*{Truck egress (Figure~\ref{res_truck_pres} and Figure~\ref{res_truck_bd}) -- Humanoid and insectoid robots.}
%\subsubsection*{Truck egress -- Humanoid and insectoid robots (Figure~\ref{res_truck_bd}).}
\subsubsection{HRP-2 -- Steep staircase (Figure~\ref{fig:stair_robust}):}

\begin{figure*}
  \centering
  \includegraphics[width=0.5\linewidth]{figures/stair}
  \caption{
           HRP-2 in the steep stair climbing scenario. }
		   \label{fig:stair_robust}
\end{figure*}

The goal is to climb three 20-cm high steps. This height requires HRP-2 to use a ramp to perform the task.

\noindent\textbf{Contacts involved:} Feet and right arm.

\noindent\textbf{Heuristics:} The extended manipulability $h_w$ is chosen for the feet; $h_{EFORT}$ is chosen for the right arm.
Regarding equilibrium, the video demonstrates two sequences computed for two different threshold values of $b_0$: $0$ and $2$ (Figure~\ref{fig:stair_robust}). 

\noindent\textbf{Observations:}
This scenario illustrates best the importance of the equilibrium-robustness criterion.
With a robust approach, more states are required to reach the last step (15 rather than 13 in average).
However, when the last step is reached by both feet, in the nonrobust case the contacts are extremely close to 
the cone limits (Figure~\ref{fig:stair_comp}).


The geometry of the environment is easily addressed by our planner, and the contact planning is several times faster than real time in this scenario.

Again, the interpolation motion between the contact stances is out of the scope of this paper. However it should be noted that the computed plan in this scenario has been executed successfully on the robot~\citep{Carpentier2016}.

\begin{figure}
  \centering
  %~ \includegraphics[width=0.6\linewidth]{figures/stair_robust}
  \begin{overpic}[width=0.5\linewidth]{figures/stair_robust}
		\put (17,5) {\small{\color{red}$b_0 = 0.23$}} 
		\put (79,5) {\small{\color{green}$b_0 = 6.16$}} 
		%~ \put (68,58) {3.a)} 
		%~ \put (5,27) {3.b)} 
		%~ \put (37,27) {4.a)} 
		%~ \put (68,27) {4.b)} 
	\end{overpic}
  \caption{
           Evaluation of the robustness $b_0$ of two contact configurations. Although in equilibrium, the left configuration is on the verge of slipping.}
		   \label{fig:stair_comp}
\end{figure}

\subsubsection{HRP-2 -- Standing up (Figure~\ref{fig:standing}):}
From a bent configuration, a standing-up motion is computed in a cluttered environment.
The resulting motion involves using a wall as support, and climbing a 25-cm high step.

\begin{figure*}
  \centering
  \includegraphics[width=0.5\linewidth]{figures/standing}
  \caption{
           HRP-2 in the standing scenario. }
		   \label{fig:standing}
\end{figure*}


\noindent\textbf{Contacts involved:} All (both feet and hands).

\noindent\textbf{Heuristics:} $h_w$ for the feet, $h_{EFORT}$  for the hands.

\noindent\textbf{Observations:} The scenario illustrates well the acyclic aspect of the planning. For instance, in the four first frames of Figure~\ref{fig:standing}, we can see that the right foot
is moved twice, with the left foot in between, before the configuration allows HRP-2 to move its hand.
Because the contacts are tried in a FIFO manner, the fact that the output contact sequence is acyclic shows that a cyclic approach (with a finite state machine for instance) is not sufficient
for the computed path. The reason for this is not reachability, but equilibrium. The planning is slower than for the stair scenario (because the contact generation fails more),
though it remains compatible with interactive performances. % \adnote{Maybe define what you mean by 'interactive' and 'real-time' at some point.}

%~ \subsubsection{HRP-2 -- Truck egress (Figure~\ref{fig:TODO}):}
%~ HRP-2 stands half seated in a truck, and \replaced{is required to}{it must} \replaced{get out}{exit} through the front window.

%~ \begin{figure*}
  %~ \centering
  %~ \includegraphics[width=0.5\linewidth]{figures/standing}
  %~ \caption{
           %~ HRP-2 in the standing scenario. }
		   %~ \label{fig:standing}
%~ \end{figure*}

%~ 
%~ \noindent\textbf{Contacts involved:} All (both feet and hands).
%~ 
%~ \noindent\textbf{Heuristics:} $h_w$ \deleted{is used} for the feet, $h_{EFORT}$ \deleted{is used} for the hands.
%~ 
%~ \noindent\textbf{Observations:} This scenario demonstrates the ability of the planner to compute a path in an extremely-constrained environment. In this case,
%~ the scenario is not always real time (TODO or is it ?): while the planner is often able to compute a guide path rapidly, in the worst case up to one minute can be spent
%~ in the planning.


\subsubsection{HyQ -- Darpa-style rubble (Figure~\ref{fig:darpa})}
The quadruped robot is given the task to cross a rubble composed of bricks rotated at different angles and directions.

\begin{figure*}
  \centering
  \includegraphics[width=0.5\linewidth]{figures/darpa}
  \caption{
           Robust crossing of rubbles by HyQ ($b_0 > 20$). }
		   \label{fig:darpa}
\end{figure*}


\noindent\textbf{Contacts involved:} All (the 4 legs).

\noindent\textbf{Heuristics:} $h_w$ for all legs. The robustness threshold $b_0$ is set to $20$.

\noindent\textbf{Observations:} In this context, setting up a really important minimum value for $b_0$ is possible due to the high
stability of the HyQ robot, and results in more contact switches, in exchange for safety. The path-planning time is higher than for the previous HRP-2 robots due to the constraint that the 4 reachable workspace of all legs must
collide with the environment at all times. %\adnote{Why do not relax this constraint and require only 3 legs?} 
Again, the computed time remains however interactive.

\subsubsection{HyQ -- Obstacle race (Figure~\ref{fig:hyq_bridge} and~\ref{fig:hyq_obs}):}
In this long scene, HyQ is first required to cross a 55-cm large hole; then, to cross a narrow "bridge",  only 25-cm large.

\begin{figure}
  \centering
  \includegraphics[width=0.4\linewidth]{figures/hyq_bridge}
  \caption{
           HyQ crossing a narrow bridge. }
		   \label{fig:hyq_bridge}
\end{figure}

\begin{figure*}
  \centering
  \includegraphics[width=0.5\linewidth]{figures/hyq_obs}
  \caption{
           Crossing a hole contact sequence for HyQ ($b_0 > 4$). }
		   \label{fig:hyq_obs}
\end{figure*}



\noindent\textbf{Contacts involved:} All (the 4 legs).

\noindent\textbf{Heuristics:} $h_w$ for all legs. The robustness threshold $b_0$ is set to $4$.

\noindent\textbf{Observations:} Despite the apparent simplicity of the scene, this scenario is the hardest for our planner.
While finding a guide path above the hole is easy for the planner, finding a sequence of contacts that allows for equilibrium is not trivial.
Second, the narrow bridge is hard both for the planner and the contact generator: to make sure that equilibrium is preserved along the traversal,
the bridge must be approached with the right angle.
This can is illustrated in Figure~\ref{fig:hyq_obs}, where several feet rearrangements are required to cross both obstacles (although the video shows this best).
The planner however succeeds in finding a feasible sequence in the end, again with interactive computation times.

\subsubsection{HRP-2 -- Path re-planning(Figure~\ref{fig:re-planning}):}
In this long scene, HRP-2 plans a path through several obstacles. the scene is edited during the execution of the motion: a stair is added,
some stepping stones are removed, and part of the final staircase is deleted. All these modifications require re-planning.


\begin{figure}
  \centering
  \includegraphics[width=0.7\linewidth]{figures/replanning}
  \caption{
           HRP-2 in the re-planning scenario. After the red step stones are removed, a new sequence of contacts is re-planned. Hand contacts
           are not presented here for readability.}
		   \label{fig:re-planning}
\end{figure}

\noindent\textbf{Contacts involved:} Feet and the right arm.

\noindent\textbf{Heuristics:} $h_w$  for all legs. $h_{EFORT}$  for the right arm. The robustness threshold is set to $2$.

\noindent\textbf{Observations:} This scenario is designed to illustrate concretely the computation times of the planner.
In the video, the footsteps indicating the contact sequence appear at the average speed of their computation (including the guide-path planning).


\subsubsection{3-fingered hand -- Manipulation of a pen (Figure~\ref{fig:penrot}):}
This scenario is proposed to illustrate the generality of our approach: we consider a manipulation task for a robotic hand and use
our contact planner to compute a guide trajectory for the fingers, considered as effectors (Figure~\ref{fig:penrot}).
Although we do not address the hard issue of accounting for rolling motions, the planner is able to compute the shown sequences in less than 5 seconds.

\begin{figure*}
\centering
  \begin{overpic}[width=1\linewidth]{figures/penrot}
	\end{overpic}
\caption{Contact sequence found for a pen manipulation in a zero gravity environment.}
		   \label{fig:penrot}
\end{figure*}

 
\noindent\textbf{Contacts involved:} Three finger-tips.

\noindent\textbf{Heuristics:} $h_{EFORT}$ for all fingers.
 
 
\subsection{Performance analysis} \label{sec:perf}
To analyze performance, for each considered scenario, we ran the simulation 1000 times.
We measured the computation time spent in each aspect of the algorithm, and also analyzed the success
rate obtained for each scenario.

\subsubsection{Computation times}
Table~\ref{tab:requestime} summarizes the performance measurement obtained, in terms of computation times.

\begin{table*}
\centering
\small
\begin{tabular}{ l | >{\centering\arraybackslash}m{57pt} | >{\centering\arraybackslash}m{70pt} | >{\centering\arraybackslash}m{75pt} | >{\centering\arraybackslash}m{77pt} | >{\centering\arraybackslash}m{85pt} | c}
  &  Complete guide generation & Static equilibrium & Collision & ik  & Total generation time & 1 step\\
 \hline
   Stairs & 6 -- \textbf{13} --  24 & 13 --  \textbf{32} -- 329   & 1 --  \textbf{4} -- 38 & 26 --  \textbf{127} -- 1345 & 92 --  \textbf{261} -- 2174 & \textbf{14} \\
   Standing & 4 -- \textbf{451} --  985 & 27 --  \textbf{144} -- 338   & 2 --  \textbf{12} -- 37 & 144 --  \textbf{1046} -- 2374 & 310 --  \textbf{1622} -- 3429 & \textbf{66} \\
   Rubble & 3 -- \textbf{524} --  993 & 242 --  \textbf{511} -- 3480   & 233 --  \textbf{505} -- 4564 & 180 --  \textbf{414} -- 3518 & 1400 --  \textbf{3058} -- 13126 & \textbf{43} \\
   Race & 1 -- \textbf{501} --  995 & 266 --  \textbf{449} -- 3956   & 824 --  \textbf{1061} -- 2130 & 666 --  \textbf{874} -- 1613 & 2530 --  \textbf{3722} -- 10030 & \textbf{33} \\
   %Standing up & $70$ & 23 & 48\\
 \end{tabular}
\caption{minimum, \textbf{average} and worst time (in ms) spent in the generation process for each scenario and each critical part of the generation process. The last
column indicates the average time necessary to compute one contact transition.}
\label{tab:requestime}
\quad
\end{table*}

In most scenarios, one can observe that the average computation time for one single step is largely below the second,
thus allowing to consider interactive applications. 
Considering the repartition of the computation time, it appears that most of the time is spent performing inverse kinematics.
This is not surprising considering the number of calls to the methods: IK is used intensively to maintain contact continuity between two postures; 
it is also applied every time a new candidate needs to be evaluated. However regarding state-of-the-art IK solutions, we believe there is room
for improving the performances of the algorithm.

\subsubsection{Success rates}
Table~\ref{tab:requestpercent} summarizes the success rates obtained for each scenario.
\begin{table*}
\centering
\begin{tabular}{ l | >{\centering\arraybackslash}m{65pt} | >{\centering\arraybackslash}m{65pt} | >{\centering\arraybackslash}m{65pt} | >{\centering\arraybackslash}m{65pt} | c}
  &  Path planning & True feasibility & Kinematic failure & Equilibrium failure \\
 \hline
   Steep stairs & 100\%  & 99.5\% & 0.11\% & 0.39\% \\
   Standing up & 68\% & 88\% & 6\% & 6\% \\
   %~ Truck egress & \% & \% & \% &  \% \\
   Rubble & 74\% & 97.87\% & 0.13\% & 2\% \\
   Obstacle race & 58\% & 95.7\% & 1.8\% & 2.5\% \\
 \end{tabular}
\caption{Success rates for each scenario. The first value indicates the percentage of successfull complete contact plannings for 1000 tests; The second value
indicates the percentage of truly feasible root configurations: considering each limb individually, indicates the percentage of root placements of the guide trajectory that
led to a feasible contact. Kinematic failure is the percentage of contact generations that failed because no collision-free candidate was found. Equilibrium failure is the percentage of contact
generations that failed because no candidate respected the static equilibrium condition.}
\label{tab:requestpercent}
\quad
\end{table*}

We observe that, as expected, our planner does not succeed systematically, because of the approximations made in the implementation
of our theoretical formulation. 
From a pragmatic point of view, regarding the computation times, we claim that our approach is the best current compromise between completeness and efficiency:
Indeed, the advantage of the framework is that when the contact generation fails, it does so rapidly, which allows us to rapidly re-plan a new contact sequence with 
a reasonable chance of success.
