We present a framework capable of producing contact plans describing complex multiped motions: standing up, climbing stairs using a handrail, crossing rubble and getting out of a car. The need for such a planner was highlighted at the Darpa Robotics Challenge, where required multi contact behaviors
could not be demonstrated.

Indeed, current planners suffer from prohibitive computation times, explained by the difficulty
of generating contact postures: the feasible space (the contact manifold) is of high dimension,
and impossible to sample.

We tackle this issue with our novel key idea, the reachability condition. This condition allows us to define
a geometric approximation of the contact manifold, which is of low dimension and can be sampled efficiently.
Informally, the condition verifies that the root configuration of a robot is “close, but not too close” from obstacles: close to allow contact creation, not too close to avoid collision. Thanks to this condition we decompose the hard contact planning problem into
simpler sub-problems: first, to plan a guide path for the root without considering the whole-body configuration, using a sampling
based algorithm; then, to generate a discrete sequence of whole-body configurations in static equilibrium along this path, using a new deterministic
contact selection algorithm. 

Our approach results from the pragmatic choice of favoring efficiency over completeness, justified empirically: in a
few seconds, with satisfying success rates, we generate complex contact plans for various scenarios and robots, namely
HRP-2, HyQ, and a dexterous hand. These plans are then validated in dynamic simulations, or on the real HRP-2 robot, thus demonstrating 
the first interactive multi-contact planner.
