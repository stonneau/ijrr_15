We present a framework capable of planning 
a large variety of contact plans, which describe complex multiped
motions (including humanoid) such as standing up, as climbing stairs using a handrail, crossing rubble and getting out of a car.
As this choice of scenarios suggests, our framework answers a need demonstrated at the Darpa Robotics Challenge,
where the lack of an automatic acyclic contact planner was recognized as a major issue. 

The key idea of the framework is the introduction of the reachability condition.
Informally, it verifies that the root configuration of a robot is ``close, but not too close'' from the obstacles of the environment: 
close enough to allow contact creation, but not too close to avoid collision.
This approximation of the space of admissible root configurations allows the decomposition of the high-dimensional, combinatorial contact planning problem into two simpler sub-problems:
first, to plan a guide path for the root of the robot without considering the whole-body configuration	 of the robot; then, to generate a discrete sequence of whole-body configurations in static equilibrium along this path. The reachability condition turns the computation of the guide path into a low dimensional collision checking problem, which can be solved in less than a few seconds.
Then, a deterministic contact selection algorithm tackles the combinatorial aspect of the second problem, the generation of a discrete sequence along the guide path.
Several innovations make it computationally efficient, including a criterion for verifying the static equilibrium of a configuration, and a set of heuristics used 
to enforce desirable properties on the configuration.

The design of our planner results from the pragmatic choice of favoring efficiency over exhaustiveness, which we justify empirically:
in a few seconds, with satisfying success rates, we are able to generate complex contact plans for a large variety of scenarios and robots, including HRP-2, HyQ, and a dexterous hand.
