We describe  a method to plan contacts allowing any multiped robot, including a humanoid, to achieve complex motions: standing up, climbing stairs using a handrail, crossing rubble and getting out of a car. 

The Darpa Robotics Challenge recognized the lack  of an automatic acyclic contact planner to be a major issue. We adress by introducing the reachability condition which, informally, verifies that a robot root remains in a bounded volume above the environment surfaces, close enough to allow contact creation and far enough to allow collision free motion of the limbs.  We achieve a complex motion by decomposing the hard contact planning into simpler sub-problems: first, to plan a guide path for the root without considering the whole-body configuration; second, to generate a discrete sequence of whole-body configurations in static equilibrium along this path. The reachability condition turns the high-dimensional computation of the guide into a collision checking problem. Then a deterministic contact selection algorithm tackles the combinatorial issue of  generating of a discrete sequence along the guide path. Several innovations make it computationally efficient: a criterion for verifying static equilibrium, and a set of heuristics used to enforce desirable properties on the configuration.

Our approach results from the pragmatic choice of favoring efficiency over exhaustiveness, justified empirically: in a few seconds, with satisfying success rates, we generate complex contact plans for various scenarios and robots, namely HRP-2, HyQ, and a dexterous hand.
