% !TEX root =  ../main.tex
We present a contact planner for complex legged locomotion tasks: standing up, climbing stairs using a handrail, crossing rubble and getting out of a car. The need for such a planner was shown at the Darpa Robotics Challenge, where such behaviors
could not be demonstrated (except for egress).
%
%This planning problem is theoretically known to be a specific instance of a larger class of complex planning problems (``manipulation'' or ``rearrangement'' planning).
Current planners suffer from prohibitive algorithmic complexity, because they deploy a tree of robot configurations projected in contact with the environment.
Contact configurations are costly to compute (because sampled and projected in high dimension) and have no equivalence property (i.e. two neighbor configurations cannot be directly connected in the planning graph).
%
We tackle this issue by introducing a reduction property: the reachability condition. 
This condition defines a geometric approximation of the contact manifold, which is of low dimension, has Cartesian topology and can be efficiently sampled and explored.
%
The hard contact planning problem can then be separated into two sub-problems: 
first, we plan a path for the root without considering the whole-body configuration, using a sampling-based algorithm; 
second, we generate a discrete sequence of whole-body configurations in static equilibrium along this path, using a deterministic contact-selection algorithm. 
%
%We tackle this issue by introducing the reachability condition. This condition allows us to define
%a geometric approximation of the contact manifold, which is of low dimension and can be sampled efficiently.
%Informally, the condition verifies that the root configuration of a robot is close, but not too close to obstacles: close to allow contact creation, not too close to avoid collision. Thanks to this condition we decompose the hard contact planning problem into two
%sub-problems: first, planning a guide path for the root without considering the whole-body configuration, using a sampling-based algorithm; then, generating a discrete sequence of whole-body configurations in static equilibrium along this path, using a deterministic contact-selection algorithm. 
%
The reduction breaks the algorithm complexity encountered in other contact planners, logically leading to the open-source implementation of the first interactive multi-contact planner.
%
While no contact planner has yet been proposed with theoretical completeness, we empirically show by a large number of trials ($>$10,000) that it is not an issue for our method:
in a few seconds, with high success rates, we generate complex contact plans for various scenarios and robots: HRP-2, HyQ and a dexterous hand. 
These plans are then validated either in dynamic simulations and on the real HRP-2 robot.
%, thus demonstrating the first interactive multi-contact planner.
%Our approach results from the pragmatic choice of favoring efficiency over completeness, which we justify empirically
