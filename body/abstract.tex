% !TEX root =  ../main.tex
We present a contact planner for complex legged locomotion tasks: standing up, climbing stairs using a handrail, crossing rubble and getting out of a car. The need for such a planner was shown at the Darpa Robotics Challenge, where such behaviors
could not be demonstrated (except for egress).

Current planners suffer from their prohibitive algorithmic complexity, because they deploy a tree of robot
configurations projected in contact with the environment.
%~ computation times, due to the difficulty
%~ of generating contact postures in a high-dimensional solution space that is impossible to sample.

We tackle this issue by introducing a reduction property: the reachability condition. This condition defines
a geometric approximation of the contact manifold, which is of low dimension, presents a Cartesian topology, and can be efficiently sampled and explored.
%~ Informally, the condition verifies that the root configuration of a robot is close, but not too close to obstacles: close to allow contact creation, not too close to avoid collision. Thanks to this condition we decompose 
The hard contact planning problem can then be decomposed into two
%~ sub-problems: first, planning a guide path for the root without considering the whole-body configuration, using a sampling-based algorithm; then, generating a discrete sequence of whole-body configurations in static equilibrium along this path, using a deterministic contact-selection algorithm. 
sub-problems: first, we plan a path for the root without considering the whole-body configuration, using a sampling-based algorithm; then, we generate a discrete sequence of whole-body configurations in static equilibrium along this path, using a deterministic contact-selection algorithm. 

The reduction breaks the algorithm complexity encountered
in previous works, resulting in the first interactive
implementation of a contact planner (open source). While
no contact planner has yet been proposed with theoretical
completeness, we empirically show the interest of our framework:
in a
few seconds, with high success rates, we generate complex contact plans for various scenarios and two robots, HRP-2 and HyQ. These plans are validated either in dynamic simulations, or on the real HRP-2 robot.

%~ Our approach results from the pragmatic choice of favoring efficiency over completeness, which we justify empirically: in a
%~ few seconds, with high success rates, we generate complex contact plans for various scenarios and robots: HRP-2, HyQ, and a dexterous hand. These plans are then validated either in dynamic simulations, or on the real HRP-2 robot, thus demonstrating 
%~ the first interactive multi-contact planner.
