\documentclass[a4paper]{article}
 \usepackage{fullpage}
\usepackage{amsfonts} % pour les lettres maths creuses \mathbb
\usepackage{amsmath}
\usepackage[T1]{fontenc}
\usepackage[utf8]{inputenc}
\usepackage[frenchb]{babel}
\usepackage{aeguill}
\usepackage{graphicx}
\usepackage{hyperref}
\usepackage{color}
\usepackage{listings}
\usepackage{pdfpages}

\newcommand\dx{\dot{x}}
\newcommand\dy{\dot{y}}
\newcommand\ddx{\ddot{x}}
\newcommand\ddy{\ddot{y}}
\def\real{{\mathbb R}}
\frenchbsetup%
{%
StandardItemLabels=true,%
ItemLabels=\ding{43},%
}%

\author {}
\title {Cover letter for the IJRR submission \\ ``An efficient acyclic contact planner for multiped robots''}
\date {}
\begin{document}
\maketitle

Dear editor and reviewers, \\ \\

The current submission is an extension to our conference paper ``A reachability-based planner for sequences of acyclic contacts in
cluttered environments'', accepted to ISRR 15 (\url{http://stevetonneau.fr/files/publications/isrr15/isrr15.pdf}).

In these two papers, our objective is to propose automatic methods
to compute the contact sequence that characterizes a robot motion in
cluttered environment.

We joined to the present submission the reviews that were made on the
conference paper. They follow the cover letter.

The extension of our work addresses the main comments of the reviewers:

\begin{itemize}
\item They were concerned by the fact that our demonstrations were
considering virtual avatars and not actual robots. In the present
submission, all our examples are applied to the HRP-2 and HyQ robots (Section 6).
To achieve this, a new contribution is proposed in our paper:
We introduce a robustness criterion for asserting the static equilibrium
of a configuration (Section 5.1). We use it in combination with new heuristics that we
propose to bias the contact generation process towards configurations
relevant for the motion being performed (Section 5.2).
With the exception of the dextrous hand, all the demonstrations shown in
the paper and the video are entirely new.

\item They also judged that the conference paper provided a theoretical
description of our approach, but lacked the details on how to apply
the method on an actual robot. We now explain the complete process
for the HRP-2 in the present submission (Appendix A).

\item Concerning remarks made on the reproducibility of the method,
we now provide the complete pseudo code of the algorithm, as well as the
source code of our application (Appendix B).

\item The paper has been largely re-written: we provide a clearer
comparison to the state of the art (Section 1.1); we clarify our statement regarding
our computation times (Section 6 and 6.1.6); we make a clear distinction between the
theoretical concepts we propose and their approximation (Section 2).

\item Regarding the computation time performances, compared to our conference paper, the planner is much more efficient.
A fundamental change is the reimplementation of the algorithm as a Bi-RRT planner instead of the PRM
approach. The code is also more efficient thanks to its complete rewriting into the HPP platform.

\item Finally, we propose an analysis of the performances of our method and
the success rates we obtained for each scenario (Section 6.2). This analysis provides
a support for our argumentation on the strong choices made in the design
of our planner (Section 1.2 and 2).
\end{itemize}

We are convinced that these points not only provide a complete answer to
the reviews of the conference paper, but also bring new contributions to
the community, that justify a journal publication.

%~ On a last note, as required by the IJRR submission form,
%~ we proposed a list of editors and reviewers for the submission.
%~ We affirm that we don't have conflicts of interest with the people we propose.
%~ For complete clarity however, we want to mention that some of the authors
%~ are proposing two workshops for the ICRA conference this year.
%~ Among the names we proposed for reviews, Karen Liu, kris Hauser, and Quang-Cuong Pham have accepted to be invited speakers.

We thank the editor and the reviewers for their time.


Sincerely, \\

The authors

\includepdf[pages=-]{reviewisrr.pdf}


\end{document}
