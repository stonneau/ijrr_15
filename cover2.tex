\documentclass[a4paper]{article}
 \usepackage{fullpage}
\usepackage{amsfonts} % pour les lettres maths creuses \mathbb
\usepackage{amsmath}
\usepackage[T1]{fontenc}
\usepackage[utf8]{inputenc}
\usepackage[frenchb]{babel}
\usepackage{aeguill}
\usepackage{graphicx}
\usepackage{hyperref}
\usepackage{color}
\usepackage{listings}
\usepackage{pdfpages}
\usepackage{longtable}

\newcommand\dx{\dot{x}}
\newcommand\dy{\dot{y}}
\newcommand\ddx{\ddot{x}}
\newcommand\ddy{\ddot{y}}
\def\real{{\mathbb R}}
\frenchbsetup%
{%
StandardItemLabels=true,%
ItemLabels=\ding{43},%
}%

\author {}
\title {Cover letter for the IJRR submission \\ ``An efficient acyclic contact planner for multiped robots''}
\date {}
\begin{document}
\maketitle

Dear Editor, \medskip

First of all we thank you for the time spent on considering our original manuscript. Indeed,
%This cover letter for the manuscript ``An efficient acyclic contact planner for legged robots'' plays more the role of a rebutal, since the paper is a resubmission for the ISSR'15 sepcial issue, and that we ask for reviewer continuity.
this cover letter introduces the first revision of the manuscript ``An efficient acyclic contact planner for legged robots'', resubmitted for the ISSR'15 special issue.
As discussed with Prof. Hollerbach, we hope to have reviewer continuity for the revision.
The cover letter then plays more the role of a rebuttal. We left the original cover letter at the end of this document, as well as the original ISRR reviews.
Therefore, we kindly ask that this cover letter be forwarded to the reviewers of the new manuscript.

We would then like to thank the reviewers for their previous feedback on the initial manuscript. We believe we have answered
all of their concerns, the main ones being that our method had not been tested on a real robot or a simulation, and that the paper was too verbose.
In the following lines we are going to describe how we improved our paper in this regard, and answer to the other remarks of the reviewers.

\section{Recall of main paper contributions}
The paper is concerned with the problem of planning acyclic contact movements. 
More precisely, we propose a very efficient method to address what we believe is the core of the difficulty of the contact planning problem: discovering the sequence of key contact postures that would lead to a safe and stable contact trajectory of the robot. 
Using the terminology of the paper, our contact planner addressed the two sub-problems $\mathcal{P}_1$ (finding a guide path for the robot root) and $\mathcal{P}_2$ (finding a sequence of stable contact postures and corresponding contact locations).
Our claim is that the solution of our planner leads to a straight-forward resolution of the last sub-problem $\mathcal{P}_3$ (finding the complete trajectory connecting the contact postures).
In particular, our method proposes interactive performances, i.e. the planner needs less time to compute the solution than what is needed for the robot to execute it.
This practical result corresponds to a significant improvement of the state of the art, for example regarding the capabilities of acyclic contact planners demonstrated by the humanoid teams during the Darpa Robotics Challenge last year.

Regarding this contribution, the reviews mostly criticized the structure of the paper (too verbose) and the unjustified claim that the contact sequence output by the planner leads to a feasible resolution of $\mathcal{P}_3$.

In the revision, we implemented a solver inspired by the state of the art to solve $\mathcal{P}_3$ and empirically proved the claim that our $\mathcal{P}_1$-$\mathcal{P}_2$ planner leads to easily solving $\mathcal{P}_3$. 
All the robot contact sequences reported in the paper have been converted to full contact trajectories using this $\mathcal{P}_3$ solution, that were validated in an contact simulator.
We also executed one of them on the real robot.
This extension, along with the correction of the other remarks by the reviewers, have been integrated in the revision paper, that we also reworked to make it clearer, shorter and sharper (the core of the paper is now 13 pages long, as opposed to 17 before) than the original manuscript.

The details of these modifications follow.

\section{Validation of the method (addressing  $\mathcal{P}_3$) }
In his review, Reviewer 2 stated that our plans are likely to be not executable, even in simulation.

\begin{quote}
  \textit{A quick glance at the sequences displayed in the video reveals that the "motion patterns" (and I understand these are not "motions" per se, but "contact plans") are not physically consistent (even reagrding them as mere contact plans and not as the final motions). The robot moves a contact without needing to readjust the posture in order to release that contact (or "set it free", by nullifying the contact forces on that contact). This results in sequences in which the contacts seem to be "slided" rather than removed and repositioned. This denotes a significant flaw in the algorithm.   }
\end{quote}

We were really surprised by the remark, in particular because we had already validated some of our plans.
The stair climbing scenario has been demonstrated on the real HRP-2 robot, while the standing up had been demonstrated in simulation. 
While it was mentioned in the paper, we did not insist on these demonstrations because our main contribution focuses on the other problems. 
We believe that, as Reviewer 1 notices, this observation was due to the fact that we did not clearly define what we mean by a contact plan (i.e. a discrete sequence of key contact posture with the corresponding contact locations).
We thank the reviewers for these comments, since it pointed out that what we believed to be an evidence must be carefully explained and justified.

A rigorous definition of a contact plan is now detailed in section 5.1. 

%In any case, the validation videos joined to the submission should now establish that our plans are in indeed feasible, even without considering the torque constraints.

We also implemented a solver inspired by the state of the art to dynamically interpolate the contact plan and produce a complete trajectory of the robot, that we validated in a full-dynamics simulator. 
This implementation completely validates that the contact plan is feasible by the real robot.
We also indeed executed one of the computed trajectory (climbing the stairs with HRP-2).

We have joined the full-dynamics simulation and real life videos of all the scenarios we demonstrate.
These videos show that the plans we compute correspond to feasible motions.
While we briefly describe how we solve $\mathcal{P}_3$, the focus of the paper remains $\mathcal{P}_1$ and $\mathcal{P}_2$. 
The solution of $\mathcal{P}_3$ is reported in the appendix to emphasize the main focus.

\newcommand{\mP}[1]{$\mathcal{P}_#1$}

\section{Success rates of the method}
At this point, we believe it is important to clarify our position regarding the concerns of both reviewers about the success rates of our paper.
In the paper, we are providing the success rates of our contact planner. 
Such an empirical study has yet never been provided for other contact planners.
Our method is not always successful, as very often with probabilistic searches.
More precisely, the hypothesis taken by \mP1 may lead to an unfeasible \mP2 problem, and similarly \mP2 hypotheses may lead to an unfeasible \mP3.
We accept this fact and argue that it is mostly successful (i.e. more than 77\% of success in reasonable ``DRC-like'' scenarios).
Even in case of failure, the planner is based on random sampling that approximate probabilistic completeness, and can be restarted to obtain another candidate trajectory.
In practice, it can always find a feasible solution fast enough for online applications by re-planning upon failure.

Reading the papers reporting previous works in contact planning, it is in fact not discussed whether the methods are always successful or not. We simply have no idea. For instance, the CIO of Mordatch,
because of the local convergence of the method, \textbf{necessarily} fails in some scenarios. We just do not know how often. Similarly, Escande et al. had to manually edit 
the end effector trajectories to avoid collisions with the environment, but we do not know how often. Considering this,
we do not understand why we are criticized for providing our success rates, which actually demonstrate the interest of our approach.

\section{Originality and interest of the work}
We strongly disagree with Reviewer 2 regarding the originality of the work when he states that we simply ``\textit{experiment with minor heuristic adjustments}''.
Bretl formalized well the problem without providing an algorithm applicable to the general case. Escande focused on the contact planning $\mathcal{P}_2$. 
Our work tackles $\mathcal{P}_1$ as well, in a much more efficient manner than Bouyarmane et al.
These works, including the one of Hauser, are limited by the computational complexity of considering the full dimension of the problem.
In practice, the underlying complexity imply that they cannot lead to implementation that can be used in practical scenarios, as emphasized by the lack of such contact solvers during the DRC (as discussed by Atkinson).

We claim that the reachability condition provides a dimensionality reduction of the problem. This is clearly an originality regarding the previous work who always consider the whole-body problem.
We believe that this reduction is the only way to make the contact planning problem tractable, in the same way that many researchers (such as ourselves, but also Wieber, Tedrake, Righetti and others) use dimensionality reduction successfully to address the trajectory optimization problem $\mathcal{P}_3$.

Furthermore, the computational time provided by our method is a contribution in itself: it makes the difference
between a method that can be used at the Darpa challenge or not, as we show in our introduction.

\section{A clearer paper}
As requested, we have deeply shortened the paper. More importantly, we have done an important work of defining in simpler terms
the different concepts that we use in the paper, and reduced their number.
We believe that the paper is much clearer in its current form.

\section{Justification of the simplifying assumptions}
Regarding several comments from both reviewers, stating that our paper does not justify the fact that contact feasibility implies equilibrium feasibility,
we now have added a section that discusses in depth the assumption (Section 7). We clearly describe which hypothesis (we call it the quasi-flat hypothesis, referring
to the contacts in the configurations, rather than the vague cluttered term) are required to validate our approach, and analyze
the success rates we obtained in each scenario to justify that the assumption works.

\section{Comparison with other methods}
As noted by Reviewer 2, in our previous paper it was indeed complex to compare our work with previous contributions,
since they were addressing the complete problem. We now give the complete computation times and compare them with previous works,
clearly demonstrating that our method outperforms them by orders of magnitude (Section 6.4).


\section{On the static equilibrium of each configuration}
Reviewer 1 stated that ``\textit{the planned contact path is only discrete with static equilibrium considered for each
configuration, which is not very realistic''}. While we agree with the observation, we first recall that this assumption is made in the previous works that address cluttered scenarios, such as the work from Bretl and Escande. Works that do not make this assumption such as Mordatch and al. only offer local convergence and cannot address these scenarios (e.g. on the ``Polaris car egress'' scenario reported in our paper). Furthermore, this limitation is in fact less restrictive in our approach that the one of Escande, since while our plan is quasi static, our resulting motion is dynamic, as now explained in the paper (Section 5.1). 

\section{Actual car model of the DARPA Challenge}
As asked by Reviewer 2, we now use the Polaris car model for demonstrating our egress scenario.

\newpage
\section{Recap of Reviewer comments}

\subsection{Reviewer 1}
\noindent
\begin{longtable}{|p{21em}|p{21em}|}
\hline
The paper addresses an important problem of planning robot motion involving contacts. It extends the idea of planning acyclic contacts by decoupling the issue into (1) planning a guide path for the root of the robot, and (2) planning a discrete sequence of acyclic and static equilibrium contact configurations along the path, with an interpolation of motion between two consecutive contact postures in the sequence. The introduced planner considers both collision avoidance and maintenance of desired contacts and aims at real-time or interactive performance. 
&Thanks for the comment.
\\ \hline % ----------------------------------------------------------------------------------
For the decoupled approach to work, the planned guide path must guarantee that a sequence of stable contact configurations is associated with it to execute the path with desired contacts and yet without undesired collisions; such a guide path is called “equilibrium feasible”. The issue is how to find such a guide path efficiently. One key assumption of the paper is that for cluttered environments, “most of the times” a guide path that is “contact feasible” (referring to that at least one end-effector is in contact with the environment and there is no undesired collision between the robot and the environment) is also “equilibrium feasible”. It further assumes that if the reachable space of the robot along the guide path is in collision with the environment but the robot root configuration is not in collision, called the “reachability condition”, then the guide path is “contact feasible”. With those assumptions, the paper treated guide path planning as a geometric problem and used an RRT planner to solve the problem. 
& We agree with this summary.
\\ \hline % ----------------------------------------------------------------------------------
However, the question is whether the above key assumption (of 
“contact feasible” implies “equilibrium feasible”) is reasonable. The paper did not provide justification for the assumption, and the assumption does not seem intuitive. The paper used terms such as “most of the times” and “cluttered contact planning” without clear and quantifiable definitions. With the above assumptions, planning becomes quite straightforward, and the planning algorithm used is rather conventional. 
&
We modified the paper to more rigorously discuss these hypotheses. 
We also believe that the main justification of their validity is through statistical tests, as reported in the result section.\newline
See Section 6 for details. 
\\ \hline % ----------------------------------------------------------------------------------
For the next problem of planning a sequence of contact configurations given a “contact feasible” root path, the paper further decomposed the problem by planning the configuration of one limb at a time, assuming the root configuration and the other limb configurations were fixed. For each limb, N sampled configurations were considered (sampled off-line), and they were indexed by the corresponding end-effector position and stored in an octree beforehand. 
Because all that was done off-line, it is not clear how N was decided. 
The on-line planning step would search through the stored limb configurations (sorted according to some heuristic), one limb at a time (it seems), to find an almost-contact configuration and create a feasible limb contact configuration out of it. Different limbs were considered in some deterministic order: the limb that was contact-free the longest was considered first, and contact configurations were generated as long as the limb could maintain contact.  It is a bit confusing when interpolated contact configurations would be added between two adjacent contact configurations. Was it “in the hope” to keep a limb in contact or change limbs in contact? 
& 
--- I THINK WE MUST DISCUSS THAT HERE ---
\\ \hline % ----------------------------------------------------------------------------------
The manipulator Jacobian for each limb was assumed full rank, and no singular configurations were considered. 
&
--- I THINK WE MUST DISCUSS THAT HERE ---
\\ \hline % ----------------------------------------------------------------------------------
The main issue with the proposed planner is that assumptions (commented earlier) and decisions were made in the planner conveniently in order to simplify the problem but without much justification, and there is no guarantee of a solution. The entire approach is heuristic and somewhat ad hoc. 
&
We agree that our approach is mostly heuristic.
Heuristics are somehow underestimated in robotics, where optimality principles are often preferred (this is not the case in other fields, like discrete optimization, data routing or multi-agent diagnosis in web applications).

The main hypotheses of the papers are: the division of contact planning in three problems; the reachability condition approximating the equilibrium condition; and the heuristics for building the contact sequence (\mP2) along the guide path (\mP1).
We believe that these main hypotheses are properly discussed, some being theoretically justified (reachability condition), the other being statistically asserted.

We agree that the method also requires some other minor choices, like the ones discussed by the reviewers. 
We do not believe that these minor technical choices are important. 
If other choices are made, similar results would likely be obtained as well.
To keep the paper as short as possible, we did not spent time explaining these choices.
Any reader that would try to re-implement our approach would find enough details to reproduce our choices.
The source code of the planner is also available (url in the paper) in case the exact implementation is needed.

Providing more details about any single technical choice would require 10 more pages, which is in contraction to the request of the reviewer to have the paper less verbose.

In the revision, we emphasized what are the key assumptions, and spent more time to justify them properly.
In the appositive, we more clearly pointed what assumptions are the results of technical choices and do not impact the solver performances.


\\ \hline % ----------------------------------------------------------------------------------
Another major limitation is that the planned contact path is only discrete with static equilibrium considered for each configuration, which is not very realistic. The paper also pointed out this limitation in the Conclusions.
&
Our contribution is on solving \mP1-\mP2 efficiently. 
We believe that this is the core of the contact planning problem, as argued in introduction.
Indeed, solving \mP3 in all the cases studied in the paper was quite straight-forward.

Clearly, dividing the complete problem in 3 subproblems will ease the solver and possibly obstruct some solutions. The question is whether this choice is a sane trade-off or not. 
We insist that this division is quite classical in the state of the art.
In particular, the only planner able to cope with this division (by Mordatch et al) only leads to local convergence.

In the new version, we asserted the realism of the contact sequences by implementing a solution to \mP3, outputing full trajectories that were tested in a contact simulator and on the real HRP-2 robot.

See Sections 2 and 6 for details.

\\ \hline % ----------------------------------------------------------------------------------
The important problem of transitioning motion from one contact configuration to the next was left out as beyond the scope of the paper. 
&
--- I THINK WE MUST DISCUSS THAT HERE ---
\\ \hline % ----------------------------------------------------------------------------------
The planner was tested in a number of diverse simulation examples, and there were related comparisons with existing work. This is commendable. However, there is no experimental result on a real robot. 
&
The contribution of the paper is on subproblems \mP1 and \mP2. 
We initially did not put robot results to emphasize this aspect and avoid confusing the reader.
Following the reviewer recommendations, we added a solver of subproblem \mP3, leading to complete movement, one of them being executed on the real robot. 
\newline See Section 2 for details.
\\ \hline % ----------------------------------------------------------------------------------
The writing of the paper can be much improved and made much more concise. The description of contact path planning in Section 4 is broken into different sub-sections, but not entirely in the order of how the sequence is generated, which is confusing. In Section 4.1, steps 3—6 talk about on-line generation, but the related decisions were introduced in Section 4.2.1 and 4.2.2, and none of those sub-sections has a title that clearly indicates the planning procedure. The paper has used unnecessary notations and symbols on the one hand, but vague languages such as “most of the time”, “cluttered problems”, “in the hope”, etc., on the other hand. Some symbols were not used consistently, for instance, m is used both to indicate mass and a direction. 
&
Thanks for carefully listing these problems. 
The paper was completely rewritten to make it shorter and more rigorous.
\\ \hline % ----------------------------------------------------------------------------------
\end{longtable}

\newpage
\subsection{Reviewer 2}
\noindent
\begin{longtable}{|p{21em}|p{21em}|}
\hline
The paper presents a method for producing contact plans for the locomotion of multiped robots. It decomposes the problem into two sub-problems : the computation of a guide path is which the robot is floating close enough to the environment, and subsequently the positioning of contacts on the environment along the guide.
& Thanks for this summary, that we agree with.
\\ \hline % ----------------------------------------------------------------------------------
My major concern about this piece of work is its lack of originality. The problem seems to have been well formalized and studied in numerous previous works (Hauser et al, Escande et al, etc.). The authors experiment with minor heuristic adjustments. I do not consider them as major contributions that warrant a publication in IJRR by the standards of the journal. The authors end up with the main claim of the paper being a reduced computation time in comparison with these previous works. As admitted by the authors themselves, even this gain in computation time is not really comparable to the previous works since the outputs are different (previous works go all the way to the motion generation, and account for different constraints).
&
We do not agree with this comment.
The main contribution of our method are the reduction hypothesis that we are using to break the complexity of contact planning.
The problem reduction, leading to pragmatic implementation, is a step-change in how the problem of contact planning should be considered.

Although we clearly acknowledge the contribution of previous works, in particular by Bretl, Hauser, Escande, Bouyarmane and their co-authors, they are all mostly nonpractical contributions.
The  planners resulting of their works are interesting to study, but can barely be used in practice, in large part due to their computational complexity. 
Even the work of Mordatch and colleagues, which is solving the problem only locally and partially (no collision and self-collision checking, partial dynamics, etc) requires minutes of computations for seconds of execution.
 
We can clearly compare the performances of our planner with previous methods (even if not as accurately than what we would like): our method outperforms them by orders of magnitude.
Even if considering our contribution only on the practical aspect of computation time,  it makes the difference between a method that can be used at the Darpa challenge or not, as we show in our introduction.

\\ \hline % ----------------------------------------------------------------------------------
The work is largely incomplete, and far from being self-sufficient. A quick glance at the sequences displayed in the video reveals that the "motion patterns" (and I understand these are not "motions" per se, but "contact plans") are not physically consistent (even regarding them as mere contact plans and not as the final motions). The robot moves a contact without needing to readjust the posture in order to release that contact (or "set it free", by nullifying the contact forces on that contact). This results in sequences in which the contacts seem to be "slided" rather than removed and repositioned. This denotes a significant flaw in the algorithm. At no point in the paper do the authors explicitly define what a "contact plan" is. Hence we do not know what is the output plan of their framework they are trying to obtain (obviously that output is not the same as that of previous works which feature the above-mentioned "releasing of contacts" property in their own definitions of "contact plans"). The authors should clearly define what a valid output plan of their framework is.
&
Once more, we cannot agree with this comment.
We believe that the comment is partly due to a problem of definition of what we mean by ``contact sequence''.
The solution to subproblem \mP2 is a discrete sequence of static posture in stable contact (as detailed in Section 2).
All the sequences reported in the paper are realistic in that sense.

In the original paper, we claimed that any of these contact sequences would directly lead to a full contact trajectory (\mP3).
However, we did not asserted this claim. 
In the new version, we implemented a solver \mP3 and validated that all the contact sequences reported in the paper are leading to full contact trajectories, whose dynamic plausibility has been checked by execution on the real robot for one of them, and in a dynamic contact simulator for the others.

See Sections 2 and 6 for details.

\\ \hline % ----------------------------------------------------------------------------------
There is no way to assess the validity of the work, as opposed to the previous works. From what I can judge from the video, the motions will not be able to be performed with a real robot nor even in physics simulation. The only way to legitimately argue with the latter statement is by providing a demonstration of the execution of the plans in simulation or with the real robots. I think non-trivial work still needs to be accomplished to "plug" the output of the presented framework into any existing motion generator or controller. If the only envisioned application of the framework is the coming future work of the authors to turn it into a valid motion controller, then I think the work would be complete only by then and its validity assessable from a scientific standpoint (i.e. not speculation) only at that stage.
&
We agree that this remark is justified by the lack of facts in our original paper.
Our claim was that any contact sequence output by our planner easily lead to full contact trajectory.
We wrongly assumed that this was an evidence when looking at the contact sequences, and neglected to prove it.

In the revision, we have implement a solve \mP3 and converted all contact sequences in full contact trajectory, validated in simulation or on the real robot.
We must insist that this validation was straight-forward, on a scientific point of view. 
As initially stated, the produced contact sequences indeed lead directly to full trajectories.

See Section 2 for details.

\\ \hline % ----------------------------------------------------------------------------------
There are many algorithm weakening assumptions. The most obvious one being the replacement of the equilibrium feasibility condition with the contact feasibility condition, then replacing the latter with the reachability condition. I understand that the authors assume that these approximations are valid, especially in "cluttered" environment. However I believe it should not be too difficult to think of counter-example problems in which each of this approximations creates a degenerate plan.
&
We agree with this remark. 
The reduction models that we propose are valid approximations in a reasonable domain of the robot environment.
Experimentally, we demonstrated that this validity domain is quite large.
It is yet possible to find counter-examples, and we cite some of them in the paper.
Our planner is only a contribution to the problem, not the ultimate solution to the contact-planning problem.

In terms of genericity, two key aspects must be considered.
First, our planner is good enough for 90\% of the situations that nowadays legged robot can meet.
Second, the general idea of reducing the contact planning problem is valid in any situations that we can imagine, and the heuristics and approximate models that we proposed can be adapted to extend the planner capabilities.
Automatic synthesis of these models is indeed a research direction that we are currently considering.

In the revision, we now justified these assumptions by theoretical results, explanations about the rationale behind the assumption, and statistical validation reported in the experimental results.
\\ \hline % ----------------------------------------------------------------------------------
The joint torque limit constraint is absent from the static equilibrium condition. This is a major approximation that does indeed speed up computation with respect to this constraint being accounted for in some of the compared works, but results in weaker, less physically valid plans (again the only way to argue with that is to show that the motions are performed in real or in simulation)
&
We agree that the torque limits are not considered by our planner.
However, although this is a hard limit on the robot, we do not agree that it plays a dramatic role in contact planning.
Indeed, the torque limits are considered in our \mP3 validation, and does not negatively impact the execution of our contact sequences.
We state and experimentally verified that considering the torque limits at the planning level is not necessary.
\\ \hline % ----------------------------------------------------------------------------------
"These two contributions were required to apply our method to real robot models", the use of "real robot" is somewhat misleading there.
&
By ``real robot'', we meant a realistic modeling of the robot, with true joint limits, collision constraint, etc. 
During rewriting, this particular sentence was removed.
We now used ``real robot'' only to discuss the experiments we did on the physical robot device.
\\ \hline % ----------------------------------------------------------------------------------
A minor remark : since the DARPA challenge is one of the motivations of the work then I do not understand why the authors did not use the actual model of the car used in the DARPA challenge and used the toy model instead ? A great part of the difficulty is removed there, numerous collisions occur with detail parts of the car (steering wheel, hand break, outside frame, etc) and contribute to the challenging aspect of the problems.
&
We used the simplified car model because the rendering seemed clearer to read.
As requested by the reviewer, we implemented the same movement on the Polaris model taken from the DRC.
Our HRP-2 robot being to small to go down the car (compared to DRC HRP-2 ``Kai'' robot), we added two steps.
\\ \hline % ----------------------------------------------------------------------------------
Finally, in the presentation department, I found the paper to be excessively long, I do not know what are the pages recommendations of the journal, but he paper is too verbose, the algorithms too partitioned into different sections and subsections. I think a major rewriting and restructuring is possible and the ideas can be conveyed in a more synthetic manner.
&
The paper was rewritten, see Section 5.
\\ \hline % ----------------------------------------------------------------------------------
Overall, I think the work is very promising, and I hope that my detailed comments and questions show my enthusiasm and interest of the subject. I understand that all the approximations and heuristics are driven by a sense of  pragmatism and practicality rather than pure theoretical soundness, which are probably legitimate motivations in their own right. However I also think that the work is still not mature enough, and will largely benefit from a validation of these approximations with a real robot or simulation experiment after the framework is complete. I encourage the authors to continue along the completion of the work.
&
It was indeed difficult to fully feel the enthusiasm of the reviewer, although we definitely appreciated his detailed comment and effort to improve the quality of the paper.
In conclusion, we truly believe that our solution answer to the main difficulty of contact planning, by building a realistic contact sequence which easily leads to the construction of a complete contact trajectory, and behind, the execution of it on the real robot.
The addition of a \mP3 solver in the revised paper experimentally justifies this claim.
The results that we demonstrated have never been reached so far by any previous works, neither in term of qualitative results, number of reported scenario, statistical analysis of the algorithms or combinatorial efficiency.
The contribution that we are reporting is not a marginal increment or a practical implementation, but a step-change in how the contact planning problem should be considered, leading to the first realistic implementation.
We hope that the revision of the paper will help the reviewer to meet our opinion.

\\ \hline % ----------------------------------------------------------------------------------
\end{longtable}

\end{document}



\author {}
\title {Original Cover letter for the IJRR submission \\ ``An efficient acyclic contact planner for multiped robots''}
\date {}
\begin{document}
\maketitle

Dear editor and reviewers, \\ \\

The current submission is an extension to our conference paper ``A reachability-based planner for sequences of acyclic contacts in
cluttered environments'', accepted to ISRR 15 (\url{http://stevetonneau.fr/files/publications/isrr15/isrr15.pdf}).

In these two papers, our objective is to propose automatic methods
to compute the contact sequence that characterizes a robot motion in
cluttered environment.

We joined to the present submission the reviews that were made on the
conference paper. They follow the cover letter.

Regarding the conference paper, this appear paper presents new contributions that we first highlight:

\begin{itemize}
\item We introduce a new criterion for asserting the static equilibrium of a robot configuration. It works for arbitrary
legged robots, and handles any kind of contact configurations and surface orientation. It is formulated as a really efficient Linear Program, and 
can be used to compute a value of robustness of the configuration. This robustness is useful for real robots applications, where many uncertainties
in the state estimation of the robot can lead to incorrect evaluations of the equilibrium.
\item We present two new heuristics that we use to bias our contact planner framework towards configurations relevant for the task being performed.
\item We present a complete algorithm for contact planning that integrates the robust equilibrium criterion.
\end{itemize}

The extension of our work also addresses the main comments of the reviewers, and proposes lesser contributions,
that we detail in the following lines.

\begin{itemize}
\item They were concerned by the fact that our demonstrations were
considering virtual avatars and not actual robots. In the present
submission, all our examples are applied to the HRP-2 and HyQ robots (Section 6).
To achieve this, a new contribution is proposed in our paper:
We introduce a robustness criterion for asserting the static equilibrium
of a configuration (Section 5.1). We use it in combination with new heuristics that we
propose to bias the contact generation process towards configurations
relevant for the motion being performed (Section 5.2).
With the exception of the dexterous hand, all the demonstrations shown in
the paper and the video are entirely new.

\item They also judged that the conference paper provided a theoretical
description of our approach, but lacked the details on how to apply
the method on an actual robot. We now explain the complete process
for the HRP-2 in the present submission (Appendix A).

\item Concerning remarks made on the reproducibility of the method,
we now provide the complete pseudo code of the algorithm, as well as the
source code of our application (Appendix B).

\item The paper has been largely re-written: we provide a clearer
comparison to the state of the art (Section 1.1); we clarify our statement regarding
our computation times and compare our method with the state of the art (Section 6); we make a clear distinction between the
theoretical concepts we propose and their approximation (Section 2).

\item Regarding the computation time performances, compared to our conference paper, the planner is much more efficient.
A fundamental change is the re-implementation of the algorithm as a Bi-RRT planner instead of the PRM
approach. The code is also more efficient thanks to its complete rewriting into the HPP platform.

\item Finally, we propose an analysis of the performances of our method and
the success rates we obtained for each scenario (Section 6.2). This analysis provides
a support for our argumentation on the strong choices made in the design
of our planner (Section 1.2 and 2), and a comparison with methods from the literature.
\end{itemize}

We are convinced that these points not only provide a complete answer to
the reviews of the conference paper, but also bring new contributions to
the community, that justify a journal publication.

%~ On a last note, as required by the IJRR submission form,
%~ we proposed a list of editors and reviewers for the submission.
%~ We affirm that we don't have conflicts of interest with the people we propose.
%~ For complete clarity however, we want to mention that some of the authors
%~ are proposing two workshops for the ICRA conference this year.
%~ Among the names we proposed for reviews, Karen Liu, kris Hauser, and Quang-Cuong Pham have accepted to be invited speakers.

We thank the editor and the reviewers for their time.


Sincerely, \\

The authors

\includepdf[pages=-]{reviewisrr.pdf}


\end{document}
