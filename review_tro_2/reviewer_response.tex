\documentclass[a4paper]{article}
\usepackage{fullpage}
\usepackage{amsfonts} % pour les lettres maths creuses \mathbb
\usepackage{amsmath}
\usepackage[T1]{fontenc}
\usepackage[utf8]{inputenc}
\usepackage[frenchb]{babel}
\usepackage{aeguill}
\usepackage{graphicx}
\usepackage{hyperref}
\usepackage{color}
\usepackage{listings}
\usepackage{pdfpages}
% Another method to track changes
% Examples of usage:
% - This is \added[id=per,remark={we need this}]{new} text.
% - This is \deleted[id=per,remark=obsolete]{unnecessary}text.
% - This is \replaced[id=per]{nice}{bad} text.
% To print the list of change use: \listofchanges
%~ \usepackage{changes}	% use this for the working version
%\usepackage[final]{changes} % use this for the final version
%~ \definechangesauthor[name={Andrea Del Prete}, color=orange]{adp}
\newcommand{\deladp}[1]{\textcolor{red}{#1}}
\newcommand{\addadp}[1]{\textcolor{green}{#1}}
%~ \newcommand{\repadp}[2]{\replaced[id=adp]{#1}{#2}}
%~ \definechangesauthor[name={Steve Tonneau}, color=blue]{st}
%~ \newcommand{\delst}[1]{\deleted[id=st]{#1}}
%~ \newcommand{\addst}[1]{\added[id=st]{#1}}
%~ \newcommand{\repst}[2]{\replaced[id=st]{#1}{#2}}
\newcommand{\gls}[1]{\textit{#1}}
\newcommand{\glslink}[2]{{#2}}
%~ \newcommand{\done}[0]{\textcolor{green}{DONE}}
\newcommand{\done}[0]{}
\newcommand{\ndone}[0]{\textcolor{red}{TODO}}

\newcommand\dx{\dot{x}}
\newcommand\dy{\dot{y}}
\newcommand\ddx{\ddot{x}}
\newcommand\ddy{\ddot{y}}
\def\real{{\mathbb R} }
\newcommand\quot[1]{\begin{quote} \underline{quote}: \textbf{#1}\end{quote}}
\newcommand\as[1]{\begin{quote} \underline{answer}: {#1}\end{quote} }
\newcommand\qt[1]{\begin{quote} \underline{added text in paper}: \textit{#1}\end{quote} \leavevmode \\ }
\newcommand\jp{ \leavevmode \\}
\frenchbsetup%
{%
StandardItemLabels=true,%
ItemLabels=\ding{43},%
}%
\DeclareUnicodeCharacter{00A0}{ }
\author {}
\title {Cover letter for the resubmission of the conditionally accepted paper ``An efficient acyclic contact planner for multiped robots''}
\date {}

%% \addtolength{\oddsidemargin}{-.6in}
%% \linespread{1.5}
%% \addtolength{\textwidth}{-1.5in}

\begin{document}
\maketitle


First of all, we would like to thank all the Reviewers and the associate editor (AE) for considering our resubmission.
We are glad to find them satisfied with our modifications, and made our best to answer to the remaining comments as well as possible.
In the following lines, we will quote all the remaining comments and answer to explain how exactly we updated the paper to address them.


\section{Answers to the associate editor}

\quot {The C++/Java-style code listings are unnecessary and do not conform
to journal standards.  The authors rewrite their pseudocode to use
rigorous mathematical notation.  Suggest replacing "object.attribute"
with "Attribute(object)", and describing the attributes of the objects
in the text.
}

\as{We rewrote Algorithm 1, 2 and 3 accordingly to the recommendations of the Associate Editor, which we thank for the comment that allowed us to spot
a typo in pseudocode. Furthermore, it also allowed us to reduce the length of the pseudocode.}
\leavevmode 
\quot {The URLs in the paper seem rather brittle and are not suitable for
archival purposes.  Two URLs point to specific branches of various
Github repositories, and another refers to the documentation of a
Blender tool.  The authors should create a static web page for their
software which is guaranteed to persist for years to come... }

\as{Thanks for the comment. In fact, the github urls are not pointing to branches, but to release versions of the code, presented as zip files, which make them permanent
links. However, we agree with the Associate Editor that regrouping all the data and information on a single web page is relevant. So, we uploaded the source code to the first author
website (\url{http://stevetonneau.fr/files/publications/isrr15/tro_install.html}), and provided all the required information to install the material in it. The user has the choice to download the code either from the github permanent link or directly from the author's website.}

\leavevmode 
\quot {... and to refer
to the underlying method used by the Blender tool.}

\as{Regarding the decimate tool from Blender, we proceeded as follows: in the paper, we mention a SIGGRAPH paper as one option to reduce mesh complexity, and mention that equivalent methods are proposed by many 3D authoring tools, such as Blender which we used. Then, in the static web page, we remind this piece of information and provide a permanent link to the version of Blender we used, along with the documentation that explains how to use the decimate tool.}

\qt{The resulting polytope can contain a very large number of faces.  A last step is thus to simplify it using an incremental decimation method[37].
Variations of this method are commonly implemented in most authoring tools. For our experiments we used the Blender decimate tool. Details of its use can be found in the static webpage associated with this paper \url{http://stevetonneau.fr/files/publications/isrr15/tro_install.html\#decimate}.\\}

\section{Answers to Reviewer 1}
%~ \ndone reecriture p3 + enlever main video

\quot{First of all, I would like to thank the authors for the significant
improvements to the paper. The contributions are much clearer, the
algorithms are better described, and I believe that my initial concerns
about the paper have been addressed. In particular, I would like to
thank the authors for the detailed inline responses.}

\as{We in turn thank the reviewer, who significantly helped us improving the paper with his comments.}

\leavevmode \\
\quot {It looks like that sentence was not actually removed from the
paper (referring to the sentence ``optimization-based approaches
only converge locally'')
}

\as{Indeed, the sentence was somehow not removed from our latest version. We apologize for that, and guarantee that this time it was indeed removed!}

\leavevmode \\
\quot {
The dual reformulation of the LP in (11) in Sect. VI is fine, but is
a perfectly standard procedure (that, in fact, many LP solvers will
perform internally automatically). It would be sufficient to mention
that the dual problem is solved without giving its full formulation.
Whether the authors include the formulation is up to them.}

\as{Thanks for the comment. Indeed, using the dual formulation is quite common for LP solvers. However, after computing the dual in our case, it turns out that some of the constraints can be solved trivially, which allows to reduce the dimensionality of the problem, resulting in (11). We experimentally verified that this form is faster to solve with respect to the one from equation (10). For this reason, we believe that equation (11) is worth mentioning, and prefer to leave it, since it was given as an option by the reviewer. Finally, we removed the reference to the Slater's conditions since we agree with the reviewer that the procedure is standard.}

\leavevmode \\
\section{Answers to Reviewer 2}
\quot{All the issues I raised in the first review have been addressed in
the revised version. I'm satisfied with most of the answers and the
changes made to the paper. In particular the novelty of the paper with
respect to the conference works is clarified. I appreciate also the new
structure of the manuscript that integrates material previously
relegated to the Appendices into the main body. Now, this work seems to
be self contained enough for a journal. }
\as{Thanks for the comment. We agree with the reviewer that the paper is better this way.}

\leavevmode \\
\quot{
1. I understand the point of view of the authors who focus on
multi-contact planning and would like to refer only briefly to previous
work in motion planning for multi-legged, statically-stable robots. In
principle that is OK, but I suggest to extend the paragraph added at
the end of Section I.A by few sentences that clarify what are the
advantages of your motion planning system with respect to these older
approaches. You are focusing on humanoids, for which these gains are
pretty obvious, but what about multi-legged robots? As an inspiration I
recommend this youtube video showing a simulated hexapod that does
something similar to your HyQ simulated example
\url{https://www.youtube.com/watch?v=Xio1TbjAK80}}

\as{TODO}
\leavevmode \\

\quot{
While the answer about the environment model is satisfying, I think
that you should no only refer to the FCL library in a footnote, but to
cite the paper that explains how it works.
}
\as{Thanks for the suggestion. The citation has indeed been included, as reference [31]TODO}
\leavevmode \\

\quot{
In Tab. V you refer to the computation time in units - seconds,
minutes and hours. If so, you should be more specific as to the
source(s) of the numbers you give in that table. I understand the fact
that it is practically impossible to have a fair comparison in that
point, and the whole community is working toward changing this
situation (as you explain). But you should write explicitly that you
took the numbers from the respective papers, or from your
re-implementation of the algorithms you consider (I think it is not the
case). Well, the papers [8], [17], [18] are quite old, and comparing
the computation times achieved on the computers available 10 years ago
and nowadays isn't convincing. 
}
\as{Thanks for the comment. Indeed, we did not reimplement the solutions, so we modified the paper to state explicitely that the times indicated are those found in the original papers. Although we agree that the times do not stand as proofs, we still believe that the order of magnitude of the time computations remains significant, because an algorithm that took hours to compute 10 years ago is unlikely to perform in less than a few seconds on a modern computer, especially since our implementation is single threaded, and does not take any advantages of recent advances such as GPU programming. In any case, the reviewer is right when he says that this section should be clarified.
Thus, we removed our over-statement ``\textit{Table V presents the computation times for these scenarios, clearly demonstrating that our approach is order of magnitude faster than previous works}'' and added the following comment in Section VIII E.}
\qt{
Table V presents the computation times for these scenarios. Since no implementation of the previous methods is available, we had no choice but to indicate the times directly taken from their corresponding papers.
In the case of older contributions such as [8], [17], [18], because of the technological progress, it appears that our results would have been more meaningful if the benchmarks were run on a modern computer. However, we believe that the order of magnitudes between the performance of each method remain significant to suggest that our method is faster.
}
\leavevmode \\

\quot{I found few typos, but I believe that these would be removed in the
final editing process. 
The issues raised above are not essentia to the quality of the
contribution, but you can consider addressing these as an attempt to
improve the paper. }
\as{Thanks again for the comment. We made a last pass on the paper and indeed found a few remaining typos (negligeable -> negligible, implicitely -> implicitly, Euclidian -> Euclidean, travelled -> traveled, convervative -> conservative). More importantly we believe that we addressed all the remaining comments of the reviewer and hope that this will further improve the paper.}
\end{document}
